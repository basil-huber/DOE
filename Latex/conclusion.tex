\section{Conclusion}

In this work, we investigated the influence of three parameters on the performance of a collision avoiding algorithm design to allow collision free navigation of autonomous aerial vehicles. We measured the performance of the algorithm in terms of the mean travel time as well as the mean jerk that would be experienced by a passenger in a vehicles during the travel.

Due to technical issues on the platform, we decided to perform the experiment in simulation.
We fitted three different models on the experimental data: A purely linear model, a linear model allowing for interactions and a quadratic model allowing for linear interactions.

By comparing the p-value of the models, the linear model has been found to be the best model for modelling both the mean travel time and the mean experienced jerk. We found that the influence of parameter A is negligible for both responses. Therefore, we adjusted the model by removing parameter A. This improves the confidence of the fit and reduces the model's complexity. The travel time can be expressed as 

\begin{figure}[h]
	\centering
	%\setlength{\figW}{0.43\textwidth}
    %\begin{subfigure}[b]{0.5\textwidth}
		\setlength{\figH}{0.23\textwidth}
		% This file was created by matlab2tikz.
%
\begin{tikzpicture}

\begin{axis}[%
width=0.951\figW,
height=\figH,
at={(0\figW,0\figH)},
scale only axis,
every outer x axis line/.append style={black},
every x tick label/.append style={font=\color{black}},
xmin=-1,
xmax=1,
xlabel={$\text{x}_\text{2}\text{ (Parameter B)}$},
every outer y axis line/.append style={black},
every y tick label/.append style={font=\color{black}},
ymin=0,
ymax=3,
ylabel={arrival time},
axis background/.style={fill=white},
axis x line*=bottom,
axis y line*=left,
legend style={at={(0.03,0.97)},anchor=north west,legend cell align=left,align=left,draw=black},
xlabel shift={-4pt},
ylabel shift={-4pt}
]
\addplot [color=red,solid,mark=asterisk,mark options={solid}]
  table[row sep=crcr]{%
-1	1.09053213545266\\
0	1.15704906703525\\
0	1.15618521078093\\
1	1.18227366966137\\
};
\addlegendentry{$\text{x}_\text{3}\text{ = -1}$};

\addplot [color=green,solid,mark=asterisk,mark options={solid}]
  table[row sep=crcr]{%
-1	1.13873531444368\\
-1	1.13769868693849\\
0.165	1.6720801658604\\
1	2.38113337940567\\
1	2.40601243953006\\
};
\addlegendentry{$\text{x}_\text{3}\text{ = 0}$};

\addplot [color=blue,solid,mark=asterisk,mark options={solid}]
  table[row sep=crcr]{%
-1	1.19989633724948\\
0	1.95317899101589\\
0	1.94920525224603\\
1	2.01883206634416\\
};
\addlegendentry{$\text{x}_\text{3}\text{ = 1}$};

\addplot [color=red,dashed,forget plot]
  table[row sep=crcr]{%
-1	0.821821551505145\\
-0.9	0.864659680103842\\
-0.8	0.907497808702539\\
-0.7	0.950335937301236\\
-0.6	0.993174065899933\\
-0.5	1.03601219449863\\
-0.4	1.07885032309733\\
-0.3	1.12168845169602\\
-0.2	1.16452658029472\\
-0.1	1.20736470889342\\
0	1.25020283749211\\
0.1	1.29304096609081\\
0.2	1.33587909468951\\
0.3	1.37871722328821\\
0.4	1.4215553518869\\
0.5	1.4643934804856\\
0.6	1.5072316090843\\
0.7	1.55006973768299\\
0.8	1.59290786628169\\
0.9	1.63574599488039\\
1	1.67858412347908\\
};
\addplot [color=green,dashed,forget plot]
  table[row sep=crcr]{%
-1	1.13870562199582\\
-0.9	1.18154375059451\\
-0.8	1.22438187919321\\
-0.7	1.26722000779191\\
-0.6	1.3100581363906\\
-0.5	1.3528962649893\\
-0.4	1.395734393588\\
-0.3	1.43857252218669\\
-0.2	1.48141065078539\\
-0.1	1.52424877938409\\
0	1.56708690798279\\
0.1	1.60992503658148\\
0.2	1.65276316518018\\
0.3	1.69560129377888\\
0.4	1.73843942237757\\
0.5	1.78127755097627\\
0.6	1.82411567957497\\
0.7	1.86695380817366\\
0.8	1.90979193677236\\
0.9	1.95263006537106\\
1	1.99546819396975\\
};
\addplot [color=blue,dashed,forget plot]
  table[row sep=crcr]{%
-1	1.45558969248649\\
-0.9	1.49842782108518\\
-0.8	1.54126594968388\\
-0.7	1.58410407828258\\
-0.6	1.62694220688127\\
-0.5	1.66978033547997\\
-0.4	1.71261846407867\\
-0.3	1.75545659267736\\
-0.2	1.79829472127606\\
-0.1	1.84113284987476\\
0	1.88397097847346\\
0.1	1.92680910707215\\
0.2	1.96964723567085\\
0.3	2.01248536426955\\
0.4	2.05532349286824\\
0.5	2.09816162146694\\
0.6	2.14099975006564\\
0.7	2.18383787866433\\
0.8	2.22667600726303\\
0.9	2.26951413586173\\
1	2.31235226446042\\
};
\end{axis}
\end{tikzpicture}%
		\caption{travel time}\label{fig:concl_time}
\end{figure}	
\begin{figure}
    %\begin{subfigure}[b]{0.5\textwidth}
    		\centering
		\setlength{\figH}{0.23\textwidth}		
		% This file was created by matlab2tikz.
%
\begin{tikzpicture}

\begin{axis}[%
width=0.951\figW,
height=\figH,
at={(0\figW,0\figH)},
scale only axis,
every outer x axis line/.append style={black},
every x tick label/.append style={font=\color{black}},
xmin=-1,
xmax=1,
xlabel={$\text{x}_\text{2}\text{ (Parameter b)}$},
every outer y axis line/.append style={black},
every y tick label/.append style={font=\color{black}},
ymin=-5,
ymax=40,
ylabel={jerk},
axis background/.style={fill=white},
axis x line*=bottom,
axis y line*=left,
legend style={at={(0.03,0.97)},anchor=north west,legend cell align=left,align=left,draw=black}
]
\addplot [color=red,solid,mark=asterisk,mark options={solid}]
  table[row sep=crcr]{%
-1	1.06763197356231\\
0	1.20482048539146\\
0	1.23116266631009\\
1	1.54848024654804\\
};
\addlegendentry{$\text{x}_\text{3}\text{ = -1}$};

\addplot [color=green,solid,mark=asterisk,mark options={solid}]
  table[row sep=crcr]{%
-1	1.09509641925902\\
-1	1.09375766084999\\
0.165	12.1805684231304\\
1	35.7252695823265\\
1	38.333221902788\\
};
\addlegendentry{$\text{x}_\text{3}\text{ = 0}$};

\addplot [color=blue,solid,mark=asterisk,mark options={solid}]
  table[row sep=crcr]{%
-1	1.07879572234055\\
0	21.9381273937669\\
0	21.7369156661796\\
1	2.02139784843536\\
};
\addlegendentry{$\text{x}_\text{3}\text{ = 1}$};

\addplot [color=red,dashed,forget plot]
  table[row sep=crcr]{%
-1	-3.70436914752427\\
-0.9	-2.78821312501782\\
-0.8	-1.87205710251137\\
-0.7	-0.955901080004915\\
-0.6	-0.0397450574984664\\
-0.5	0.876410965007984\\
-0.4	1.79256698751443\\
-0.3	2.70872301002088\\
-0.2	3.62487903252733\\
-0.1	4.54103505503378\\
0	5.45719107754023\\
0.1	6.37334710004668\\
0.2	7.28950312255313\\
0.3	8.20565914505958\\
0.4	9.12181516756603\\
0.5	10.0379711900725\\
0.6	10.9541272125789\\
0.7	11.8702832350854\\
0.8	12.7864392575918\\
0.9	13.7025952800983\\
1	14.6187513026047\\
};
\addplot [color=green,dashed,forget plot]
  table[row sep=crcr]{%
-1	1.51102350983954\\
-0.9	2.42717953234599\\
-0.8	3.34333555485244\\
-0.7	4.25949157735889\\
-0.6	5.17564759986534\\
-0.5	6.09180362237179\\
-0.4	7.00795964487824\\
-0.3	7.92411566738469\\
-0.2	8.84027168989114\\
-0.1	9.75642771239759\\
0	10.672583734904\\
0.1	11.5887397574105\\
0.2	12.5048957799169\\
0.3	13.4210518024234\\
0.4	14.3372078249298\\
0.5	15.2533638474363\\
0.6	16.1695198699427\\
0.7	17.0856758924492\\
0.8	18.0018319149556\\
0.9	18.9179879374621\\
1	19.8341439599685\\
};
\addplot [color=blue,dashed,forget plot]
  table[row sep=crcr]{%
-1	6.72641616720335\\
-0.9	7.6425721897098\\
-0.8	8.55872821221625\\
-0.7	9.4748842347227\\
-0.6	10.3910402572292\\
-0.5	11.3071962797356\\
-0.4	12.2233523022421\\
-0.3	13.1395083247485\\
-0.2	14.055664347255\\
-0.1	14.9718203697614\\
0	15.8879763922678\\
0.1	16.8041324147743\\
0.2	17.7202884372808\\
0.3	18.6364444597872\\
0.4	19.5526004822936\\
0.5	20.4687565048001\\
0.6	21.3849125273065\\
0.7	22.301068549813\\
0.8	23.2172245723194\\
0.9	24.1333805948259\\
1	25.0495366173323\\
};
\end{axis}
\end{tikzpicture}%
		\caption{jerk}\label{fig:concl_jerk}
	%\end{subfigure}
	%\caption{hello}\label{fig:concl_fit}
\end{figure}

Fig. \ref{fig:concl_time} shows the experimental values for the mean travel time and the linearly interpolated data. The linear fit for the jerk can be seen in fig. \ref{fig:concl_jerk}. It can be seen that the fit for the travel time is more accurate than for the jerk. This is encouraged by a comparison of the p-value. While the travel time model has a p-value of 1.78e-8, the jerk model as a p-value of 0.0172. While the model for the travel time has an acceptable p-value, the reliability of the model for the jerk is less convincing.

We conclude that parameter A is negligible for both the travel time and the jerk. Furthermore, both responses are assumed to depend linearly on parameters B and C.