\section{Introduction}

With a high use of a shared environment by many autonomous entities, such as Personal Aerial Vehicles sharing a portion of the sky or boats near a harbor, the risk of collisions increases quadratically as function of the density \cite{jardin_analytical_2005}. Therefore there is a strong need for collision avoidance strategies in those environments. When humans are on-board, safety and comfort become main objectives of the design process. 

With current overuse of the road transportation system and planned increase in traffic\cite{truman_out_2007}, innovative solutions that overcome environmental and financial cost of the current system should be assessed. A promising idea is the use of the third dimension for personal transportation. Therefore, the European funded project myCopter\cite{jump_mycopter:_2011}\footnote{\url{http://mycopter.eu/}} aims to enable the technologies for Personal Aerial Transport Systems (PATS) as breakthrough in $21^{st}$ century transportation systems. 

We are interested in the problem where multiple decision making entities, later called agents, travel with humans on-board in a shared dense environment. As the number of conflicts rises quadratically with the number of agents \cite{jardin_analytical_2005}, the biggest issue is the risk of collisions. With humans on board, one should guarantee safety and maximize the comfort. 

%In presence of multiple decision making entities further called agents, in a given environment, the number of conflicts rises quadratically with the number of agents \cite{jardin_analytical_2005}. The biggest issue is the risk of collisions. 
% (i.e. the density)

%Having multiple agents sharing the same environment where take place interactions with humans, raises issues about comfort and safety at the same time. It should be proven that interactions are safe and the payload (humans or goods) are kept as comfortable as possible. 

Autonomous navigation has received a lot of attention for many years. With state-of-the-art techniques, it is possible to navigate safely in a known or unknown environment\cite{scherer_flying_2008}, with moving obstacles (decision making entities that cooperates or not\cite{nordlund_probabilistic_2011}) and even among humans\cite{guzzi_human-friendly_2013}. However, little focus has been made on the comfort of the passengers. Comfort is usually implemented as path planning optimization process where parameters such as the time integral of the square of the jerk are minimized \cite{gulati_framework_2009} \cite{morales_human-comfortable_2013}. These approaches work as long as the agents have a global knowledge of their environment or have enough time to recompute a usually computationally expensive optimal solution at every time step (or quite often). In dynamic environments with local (incomplete) knowledge, path planning approaches usually fail to provide safe and comfortable trajectories as the situation might change faster than the time needed to compute a new path. New ways to implement comfortable trajectories in a reactive way should be addressed. 

Comfort studies in commercial aviation focus on on-board parameters such as noise, seats size, immobility, cabin pressure and humidity \cite{hinninghofen_passenger_2006}. Other works study the effect of roll oscillation frequency on motion sickness\cite{howarth_effect_2003}, motion sickness provoked by roll and pitch motion alone or combined with vertical oscillation\cite{mccauley_motion_1976} \cite{turner_airsickness_2000} but do not propose any solution to minimize the discomfort. In addition, in a PATS, the system could reach a high density situation up to 40 vehicles per cubic kilometers\cite{truman_out_2007}. The densities usually assumed for demonstration in Air Traffic systems are many order of magnitude below what it is aimed\cite{krozel_system_2001} \cite{dowek_provably_2005}. 

%Due to smaller densities compare to our problem, Collision Avoidance strategies for commercial aviation REFNEEDED, usually assume a time before impact of many minutes. In contrary, strategies for small Unmanned Aerial Vehicles do not consider comfort as parameter since they are unmanned. 

The definition of comfort states that "\textit{comfort has both psychological and physiological components, but it involves a sense of subjective well-being and the absence of discomfort, stress or pain}"\cite{richards_psychology_1980}. In this work, we focus only on the physiological component of comfort. 

In this work, we try to find a model that allows us to estimated the influence of three parameters of the collision avoidance algorithm on the mean travel time during an experiment and the mean jerk a vehicle is subjected to. Having a good model of the travel time and the jerk will allow us to chose the parameters in order to optimize the algorithm's performance. We fit a linear model, a linear model with interactions and a quadratic model with interactions on the data. We fit the model by optimizing the effects in the sense of least-squares.
We use Analysis of Variance (ANOVA) to compare the quality of the different models.
