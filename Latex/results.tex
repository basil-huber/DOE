\section{Results}\label{sec:results}


\subsection{Linear Model}
First, we will discuss the results for the linear model.
Fig. \ref{fig:effects_lin} shows the half effects for both the travel time and the jerk in blue. It can be seen that the influence of parameter $a$ (coefficients $a_1$ and $b_1$) on both the travel time and the jerk is negligible compared to the other effects.
\begin{figure}[h]
    \centering
    \begin{subfigure}[b]{0.5\textwidth}
		\setlength{\abovecaptionskip}{1pt plus 3pt minus 0pt}	
	    % This file was created by matlab2tikz.
%
\definecolor{mycolor1}{rgb}{0.20000,0.20000,0.50000}%
\definecolor{mycolor2}{rgb}{0.00000,0.70000,0.70000}%
%
\begin{tikzpicture}

\begin{axis}[%
width=0.951\figW,
height=\figH,
at={(0\figW,0\figH)},
scale only axis,
separate axis lines,
every outer x axis line/.append style={black},
every x tick label/.append style={font=\color{black}},
xmin=0.5,
xmax=4.5,
xtick={1,2,3,4},
xticklabels={{$a_0$},{$a_1$},{$a_2$},{$a_3$},{$a_{12}$},{$a_{13}$},{$a_{23}$},{$a_{11}$},{$a_{22}$},{$a_{33}$}},
xlabel={coefficients},
every outer y axis line/.append style={black},
every y tick label/.append style={font=\color{black}},
ymin=0,
ymax=1,
ylabel={$\text{a}_\text{i}\text{/a}_\text{0}$},
axis background/.style={fill=white},
xlabel shift={-4pt},
ylabel shift={-4pt}
]
\addplot[ybar,bar width=0.5,draw=black,fill=mycolor1,area legend] plot table[row sep=crcr] {%
1	1\\
2	0.00339552585792994\\
3	0.27192029312426\\
4	0.201983314513812\\
};
\addplot [color=black,solid,forget plot]
  table[row sep=crcr]{%
0.5	0\\
4.5	0\\
};
\addplot[ybar,bar width=0.25,draw=black,fill=mycolor2,area legend] plot table[row sep=crcr] {%
1	0.998868158047356\\
3	0.273052135076903\\
4	0.201983314513812\\
};
\end{axis}
\end{tikzpicture}%
	    \caption{travel time}
	\end{subfigure}
	%\par\medskip
    \begin{subfigure}[b]{0.5\textwidth}
	    \setlength{\abovecaptionskip}{1pt plus 3pt minus 0pt}
	    % This file was created by matlab2tikz.
%
\definecolor{mycolor1}{rgb}{0.20000,0.20000,0.50000}%
\definecolor{mycolor2}{rgb}{0.00000,0.70000,0.70000}%
%
\begin{tikzpicture}

\begin{axis}[%
width=0.951\figW,
height=\figH,
at={(0\figW,0\figH)},
scale only axis,
separate axis lines,
every outer x axis line/.append style={black},
every x tick label/.append style={font=\color{black}},
xmin=0.5,
xmax=4.5,
xtick={1,2,3,4},
xticklabels={{$b_0$},{$b_1$},{$b_2$},{$b_3$},{$b_{12}$},{$b_{13}$},{$b_{23}$},{$b_{11}$},{$b_{22}$},{$b_{33}$}},
xlabel={coefficients},
every outer y axis line/.append style={black},
every y tick label/.append style={font=\color{black}},
ymin=0,
ymax=1,
ylabel={bi/bo},
axis background/.style={fill=white},
xlabel shift={-4pt},
ylabel shift={-5pt}
]
\addplot[ybar,bar width=0.5,draw=black,fill=mycolor1,area legend] plot table[row sep=crcr] {%
1	1\\
2	0.0572355129983193\\
3	0.822964208269701\\
4	0.479348852445352\\
};
\addplot [color=black,solid,forget plot]
  table[row sep=crcr]{%
0.5	0\\
4.5	0\\
};
\addplot[ybar,bar width=0.25,draw=black,fill=mycolor2,area legend] plot table[row sep=crcr] {%
1	0.980921495667227\\
3	0.842042712602474\\
4	0.479348852445352\\
};
\end{axis}
\end{tikzpicture}%
   	    \caption{jerk}
	\end{subfigure}
	
    \caption{Relative half effects on the mean travel time (a) and the mean experienced jerk (b) using a linear model without interactions; In blue results for using all parameters and in turquoise if omitting parameter $a$ (coefficients $a1$ and $b1$)}\label{fig:effects_lin}
\end{figure} 

The ANOVA table of the experiment is shown in tbl. \ref{tbl:anova_lin}. The p-value for the travel time model (2.35e-7) is significantly lower than the p-value for jerk model (4.72e-2). This suggest that the travel time model is more significant than the jerk model.

\begin{table}[h!]
	\centering
	\begin{tabular}{l l r r r r r}
Resp & Src & $\text{SS}$ & $\text{df}$ & $\text{MS}$ & $F$  &  $p$\\\hline
      Time	 & $    \alpha          	$ & 34.42	 & 4  	 & 8.606	 & 95.04	 & 2.3e-07\\
          	 & $         R          	$ & 0.815	 & 9  	 & 0.09055 \\ 
\arrayrulecolor{gray}\hline
      Time	 & $    \alpha        ^1	$ & 34.42	 & 3  	 & 11.47	 & 140.8	 & 1.8e-08\\
          	 & $         R        ^1	$ & 0.8151	 & 10 	 & 0.08151 \\ 
\arrayrulecolor{gray}\hline
      Jerk	 & $    \alpha          	$ & 2406	 & 4  	 & 601.5	 & 3.718	 & 0.047\\
          	 & $         R          	$ & 1456	 & 9  	 & 161.8 \\ 
\arrayrulecolor{gray}\hline
      Jerk	 & $    \alpha        ^1	$ & 2404	 & 3  	 & 801.5	 & 5.498	 & 0.017\\
          	 & $         R        ^1	$ & 1458	 & 10 	 & 145.8 \\ 

	\end{tabular}
	\caption{ANOVA table of the \textbf{linear model} for both responses; The $^1$ indicates the model without parameter $a$; It shows the sums of squares (SS), the degrees of freedom (df), the mean square of the error (MS), the Fisher coefficient (F) and the p-value (p), signifying the probability that this result occurs at random;}\label{tbl:anova_lin}
\end{table}

%Fig. \ref{fig:interpol_lin1} shows the experimental results for the travel time as well as the planes representing the interpolation using the effects that we have found. The colors correspond to different values of parameter C, i.e., the dots should be close to the plane of the same color.
%Fig. \ref{fig:interpol_lin2} shows the results in terms of jerk. Note that here the colors correspond to the value of the parameter A. It can be seen that the points for measurements with the same values for parameter B and C are close by, encouraging the assumption that the influence of parameter A is negligible.

%We therefore adjust both our models by omitting parameter A.
We adjust both out models by omitting parameter $a$. We expect the residue to increase slightly since we remove a degree of freedom. The removal of a degree of freedom should however increase F-factor and therefore reduce the p-value, since the residue gains a degree of freedom. The resulting estimation of the half effects is shown in fig. \ref{fig:effects_lin} in turquoise. It can be seen that the values of the half effects are similar to the previous results.

When comparing the values in the ANOVA table, it can be seen that the $p$ value decreases by a factor of approximately 13 for the travel time, suggesting that omitting parameter $a$ is improving our model. For the model of the jerk, the p-value decreases by a factor of approximately 2.7.


\subsection{Linear Model with interactions}

\begin{figure}[h]
    \centering
    \begin{subfigure}[b]{0.5\textwidth}
	    \setlength{\abovecaptionskip}{1pt plus 3pt minus 0pt}
	    % This file was created by matlab2tikz.
%
\definecolor{mycolor1}{rgb}{0.20000,0.20000,0.50000}%
\definecolor{mycolor2}{rgb}{0.00000,0.70000,0.70000}%
%
\begin{tikzpicture}

\begin{axis}[%
width=0.951\figW,
height=\figH,
at={(0\figW,0\figH)},
scale only axis,
separate axis lines,
every outer x axis line/.append style={black},
every x tick label/.append style={font=\color{black}},
xmin=0,
xmax=8,
xtick={1,2,3,4,5,6,7},
xticklabels={{$a_0$},{$a_1$},{$a_2$},{$a_3$},{$a_{12}$},{$a_{13}$},{$a_{23}$},{$a_{11}$},{$a_{22}$},{$a_{33}$}},
xlabel={coefficients},
every outer y axis line/.append style={black},
every y tick label/.append style={font=\color{black}},
ymin=-0.2,
ymax=1.2,
ylabel={ai/ao},
axis background/.style={fill=white},
xlabel shift={-4pt}
]
\addplot[ybar,bar width=0.5,draw=black,fill=mycolor1,area legend] plot table[row sep=crcr] {%
1	1\\
2	-0.000158480987423072\\
3	0.276585231712786\\
4	0.202308189537893\\
5	0.00801820250102874\\
6	-0.000745454946096567\\
7	0.116456628245764\\
};
\addplot [color=black,solid,forget plot]
  table[row sep=crcr]{%
0	0\\
8	0\\
};
\addplot[ybar,bar width=0.25,draw=black,fill=mycolor2,area legend] plot table[row sep=crcr] {%
1	1\\
3	0.273361537132869\\
4	0.202212186750112\\
7	0.116010508284771\\
};
\end{axis}
\end{tikzpicture}%
	    \caption{travel time}
	\end{subfigure}
    \begin{subfigure}[b]{0.5\textwidth}
		\setlength{\abovecaptionskip}{1pt plus 3pt minus 0pt}	
	    % This file was created by matlab2tikz.
%
\definecolor{mycolor1}{rgb}{0.20000,0.20000,0.50000}%
\definecolor{mycolor2}{rgb}{0.00000,0.70000,0.70000}%
%
\begin{tikzpicture}

\begin{axis}[%
width=0.951\figW,
height=\figH,
at={(0\figW,0\figH)},
scale only axis,
separate axis lines,
every outer x axis line/.append style={black},
every x tick label/.append style={font=\color{black}},
xmin=0,
xmax=8,
xtick={1,2,3,4,5,6,7},
xticklabels={{$b_0$},{$b_1$},{$b_2$},{$b_3$},{$b_{12}$},{$b_{13}$},{$b_{23}$},{$b_{11}$},{$b_{22}$},{$b_{33}$}},
xlabel={coefficients},
every outer y axis line/.append style={black},
every y tick label/.append style={font=\color{black}},
ymin=-0.2,
ymax=1.2,
ylabel={bi/bo},
axis background/.style={fill=white},
xlabel shift={-4pt}
]
\addplot[ybar,bar width=0.5,draw=black,fill=mycolor1,area legend] plot table[row sep=crcr] {%
1	1\\
2	0.0492031949956049\\
3	0.837675184978511\\
4	0.479930485262015\\
5	0.0189360359982001\\
6	-0.0078955424930121\\
7	0.0133129711999519\\
};
\addplot [color=black,solid,forget plot]
  table[row sep=crcr]{%
0	0\\
8	0\\
};
\addplot[ybar,bar width=0.25,draw=black,fill=mycolor2,area legend] plot table[row sep=crcr] {%
1	1\\
3	0.858420083892354\\
4	0.488671982990134\\
};
\end{axis}
\end{tikzpicture}%
   	    \caption{jerk}
	\end{subfigure}
	
    \caption{Relative half effects on the mean travel time (a) and the mean experienced jerk (b) using a \textbf{linear model with interactions}; In blue results for using all parameters and in turquoise if omitting coefficients $a_{1}$, $a_{12}$ and $a_{13}$ and $b_{1}$, $b_{12}$, $b_{13}$, and $b_{23}$)}\label{fig:effects_lin_interactions}
\end{figure} 
Fig. \ref{fig:effects_lin_interactions} shows the effects for this model. For both response variables we can again see that the influence of parameter $a$ is negligible as it was the case for the purely linear model. Furthermore, the interactions appear to be negligible in the case of the jerk model. Considering the ANOVA table for this model (tbl.~\ref{tbl:anova_lin_interactions}), it can be seen that the p-value is significantly higher than for the linear model. This suggests that the addition of the interactions does not improve the model. This is due to the fact that the number of degrees of freedom of the models increase significantly were as whereas the residues remain approximately the same as for the linear models. While removing parameter $a$ from the travel time model reduces it's p-value, it still remains superior to the linear model's p-value.

\begin{table}[h!]
	\centering
	\begin{tabular}{l l r r r r r}
Resp & Src & $\text{SS}$ & $\text{df}$ & $\text{MS}$ & $F$  &  $p$\\\hline
      Time	 & $    \alpha          	$ & 34.56	 & 7  	 & 4.936	 & 43.4	 & 0.0001\\
          	 & $         R          	$ & 0.6825	 & 6  	 & 0.1137 \\ 
\arrayrulecolor{gray}\hline
      Time	 & $    \alpha        ^1	$ & 33.62	 & 4  	 & 8.405	 & 46.75	 & 5e-06\\
          	 & $         R        ^1	$ & 1.618	 & 9  	 & 0.1798 \\ 
\arrayrulecolor{gray}\hline
      Jerk	 & $    \alpha          	$ & 2406	 & 7  	 & 343.7	 & 1.416	 & 0.34\\
          	 & $         R          	$ & 1456	 & 6  	 & 242.7 \\ 
\arrayrulecolor{gray}\hline

	\end{tabular}

	\caption{ANOVA table of the \textbf{linear model with interactions}; The $^1$ indicates the model without the coefficients $a_{1}$, $a_{12}$, and $a_{13}$;)}\label{tbl:anova_lin_interactions}
\end{table}



\subsection{Quadratic Model}

\begin{figure}[h]
    \centering
    \begin{subfigure}[b]{0.5\textwidth}
	    \setlength{\abovecaptionskip}{1pt plus 3pt minus 0pt}
	    % This file was created by matlab2tikz.
%
\definecolor{mycolor1}{rgb}{0.20000,0.20000,0.50000}%
\definecolor{mycolor2}{rgb}{0.00000,0.70000,0.70000}%
%
\begin{tikzpicture}

\begin{axis}[%
width=0.951\figW,
height=\figH,
at={(0\figW,0\figH)},
scale only axis,
separate axis lines,
every outer x axis line/.append style={black},
every x tick label/.append style={font=\color{black}},
xmin=0,
xmax=12,
xtick={1,2,3,4,5,6,7,8,9,10},
xticklabels={{$a_0$},{$a_1$},{$a_2$},{$a_3$},{$a_{12}$},{$a_{13}$},{$a_{23}$},{$a_{11}$},{$a_{22}$},{$a_{33}$}},
xlabel={coefficients},
every outer y axis line/.append style={black},
every y tick label/.append style={font=\color{black}},
ymin=-0.4,
ymax=1,
ylabel={$\text{a}_\text{i}\text{/a}_\text{0}$},
axis background/.style={fill=white},
xlabel shift={-4pt},
ylabel shift={-4pt}
]
\addplot[ybar,bar width=0.5,draw=black,fill=mycolor1,area legend] plot table[row sep=crcr] {%
1	1\\
2	0.239284478024578\\
3	0.120506889726704\\
4	0.204107368761907\\
5	-0.231907710805712\\
6	-0.000752084470361102\\
7	0.117492307258645\\
8	0.358503129231724\\
9	0.0384436095354865\\
10	-0.0964862476571698\\
};
\addplot [color=black,solid,forget plot]
  table[row sep=crcr]{%
0	0\\
12	0\\
};
\addplot[ybar,bar width=0.25,draw=black,fill=mycolor2,area legend] plot table[row sep=crcr] {%
1	1\\
2	0.239284478024579\\
3	0.120506889726704\\
4	0.204358063585361\\
5	-0.231907710805709\\
7	0.117241612435192\\
8	0.358503129231725\\
9	0.0384436095354871\\
10	-0.0964862476571693\\
};
\end{axis}
\end{tikzpicture}%
	    \caption{travel time}
	\end{subfigure}
    \begin{subfigure}[b]{0.5\textwidth}
		\setlength{\abovecaptionskip}{1pt plus 3pt minus 0pt}	
	    % This file was created by matlab2tikz.
%
\definecolor{mycolor1}{rgb}{0.20000,0.20000,0.50000}%
\definecolor{mycolor2}{rgb}{0.00000,0.70000,0.70000}%
%
\begin{tikzpicture}

\begin{axis}[%
width=0.951\figW,
height=\figH,
at={(0\figW,0\figH)},
scale only axis,
separate axis lines,
every outer x axis line/.append style={black},
every x tick label/.append style={font=\color{black}},
xmin=0,
xmax=12,
xtick={0,1,2,3,4,5,6,7,8,9,10},
xticklabels={{$b_0$},{$b_1$},{$b_2$},{$b_3$},{$b_{12}$},{$b_{13}$},{$b_{23}$},{$b_{11}$},{$b_{22}$},{$b_{33}$}},
xlabel={coefficients},
every outer y axis line/.append style={black},
every y tick label/.append style={font=\color{black}},
ymin=-2,
ymax=3,
ylabel={bi/bo},
axis background/.style={fill=white},
xlabel shift={-4pt},
ylabel shift={-5pt}
]
\addplot[ybar,bar width=0.5,draw=black,fill=mycolor1,area legend] plot table[row sep=crcr] {%
1	1\\
2	1.86406685964224\\
3	-0.227113190318953\\
4	0.544838197826409\\
5	-1.73747866012659\\
6	-0.0089633671434854\\
7	0.0151134705109152\\
8	2.77139409919224\\
9	0.181327519247342\\
10	-0.576998447029369\\
};
\addplot [color=black,solid,forget plot]
  table[row sep=crcr]{%
0	0\\
12	0\\
};
\addplot[ybar,bar width=0.25,draw=black,fill=mycolor2,area legend] plot table[row sep=crcr] {%
1	1\\
2	1.86406685964224\\
3	-0.227113190318953\\
4	0.547825986874237\\
5	-1.73747866012659\\
8	2.77139409919224\\
9	0.181327519247342\\
10	-0.576998447029369\\
};
\end{axis}
\end{tikzpicture}%
   	    \caption{jerk}
	\end{subfigure}
	
    \caption{Relative half effects on the mean travel time (a) and the mean experienced jerk (b) using a \textbf{quadratic model}; In blue results for using all parameters and in turquoise if omitting coefficients $a_{13}$ and $b_{13}$ and $b_{23}$}\label{fig:effects_quadr}
\end{figure} 


Fig.~\ref{fig:effects_quadr} shows the effects for the quadratic model. Interestingly, in this model, parameter $a$ has the largest influence on both the mean travel time and the mean experienced jerk.
%Fig. \ref{fig:interpol_quadr} shows the experimental values as well as the quadratic interpolation for the mean arrival time.
Considering the ANOVA table (tbl.~\ref{tbl:anova_quadratic}), it can be seen that the residue is smaller than for the other models. However, this is mainly due to the fact of introducing additional degrees of freedom. Furthermore, the ANOVA table shows that the p-value is higher than for the linear model, suggesting that the quality of model decreases by introducing quadratic terms.
We adjust our models by removing the coefficients $a_{13}$ from the travel time model and the coefficients $b_{13}$ and $b_{23}$ from the jerk model. While the quality of the models is increased, it is inferior to the linear models.

\begin{table}[h!]
	\centering
	\begin{tabular}{@{} l @{\hspace{8pt}} l @{\hspace{8pt}} r @{\hspace{8pt}} r @{\hspace{8pt}} r @{\hspace{8pt}} r r @{}}
Resp & Src & $\text{SS}$ & $\text{df}$ & $\text{MS}$ & $F$  &  $p$\\\hline
      Time	 & $    \alpha          	$ & 35.1	 & 10 	 & 3.51	 & 81.7	 & 1.98e-03\\
          	 & $         R          	$ & 0.129	 & 3  	 & 0.043 \\ 
\arrayrulecolor{gray}\hline
      Time	 & $    \alpha        ^1	$ & 35.1	 & 9  	 & 3.9	 & 121	 & 1.64e-04\\
          	 & $         R        ^1	$ & 0.129	 & 4  	 & 0.0322 \\ 
\arrayrulecolor{gray}\hline
      Jerk	 & $    \alpha          	$ & 3.54e+03	 & 10 	 & 354	 & 3.3	 & 1.78e-01\\
          	 & $         R          	$ & 322	 & 3  	 & 107 \\ 
\arrayrulecolor{gray}\hline
      Jerk	 & $    \alpha        ^2	$ & 3.54e+03	 & 9  	 & 393	 & 4.88	 & 7.06e-02\\
          	 & $         R        ^2	$ & 322	 & 4  	 & 80.6 \\ 

	\end{tabular}

	\caption{ANOVA table of the \textbf{quadratic} model; The $^1$ indicates the model without the coefficient $a_{13}$ and $^2$ without coefficients $b_{13}$, $b_{23}$;}\label{tbl:anova_quadratic}
\end{table}
