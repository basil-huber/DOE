\section{Procedure}

In \cite{reynolds_flocks_1987}, Reynolds first proposed a set of three simple rules to model the way birds moves in flocks. These rules are
\begin{enumerate}
 \item ``\emph{Collision Avoidance}: avoid collisions with nearby flockmates''
 \item ``\emph{Velocity Matching}: attempt to match velocity with nearby flockmates''
 \item ``\emph{Flock Centering}: attempt to stay close to nearby flockmates''
\end{enumerate}

Since then, flocking has been studied a lot in the robotic world \cite{hauert_reynolds_2011, lindhe_flocking_2005, viragh_flocking_2013}. In \cite{olfati-saber_flocking_2006}, Olfati-Saber designed a convergent flocking algorithm based on Reynolds rules. In addition to the three above mentioned rules, it takes into consideration a common objective for the whole group by introducing a migration term. This migration term attracts the agents towards their goal. We use this work in our experiments and present it here below.  

We assume that agents are influencing each other if they are closer than a distance $r$, thus we can define the set of spatial neighbor by
\begin{equation}
N_i=\{j:\|q_j-q_i\|<r\}
\label{eq:Ni}
\end{equation}

Each member of the flock is considered as an $\alpha$-agent. The goal of the flock is to create a formation where each agent is at a distant $d$ to its closest neighbors
\begin{equation}
\begin{array}{ll}
\|q_j-q_i\|=d & \forall j \in N_i
\end{array}
\label{eq:lattice}
\end{equation}

\subsection{Mathematical definitions}

We present here the mathematical tools as well as different functions used in this report and in \cite{olfati-saber_flocking_2006}. 

First, a $\sigma$\emph{-norm} is defined and replaces the standard euclidean norm which is non-differentiable at $z=0$:
\begin{equation}
\|z\|_{\sigma}=\frac{1}{\epsilon}\big[\sqrt{1+\epsilon\|z\|^2}-1\big]
\label{eq:sigmanorm}
\end{equation}
and the corresponding gradient $\sigma_{\epsilon}(z)=\nabla\|z\|_{\sigma}$:
\begin{equation}
\sigma_{\epsilon}(z)=\frac{z}{1+\epsilon\|z\|_{\sigma}}
\label{eq:sigmagrad}
\end{equation}
where $\epsilon>0$. 

Then, we define a bump function, a function that varies continuously between 0 and 1 and is null above 1:
\begin{equation}
\rho_{\delta}(z)=
\left\lbrace
\begin{array}{lll}
1 & z\in[0,\delta)\\
\frac{1}{2}\Big[1+\text{cos}\big(\pi\frac{z-\delta}{1-\delta}\big)\Big] & z\in[\delta,1]\\
0 & \mbox{otherwise}
\end{array}\right.
\label{eq:bump}
\end{equation}
with parameter $\delta\in[0,1]$.

Finally, a action function is defined as 
\begin{align}
&\phi_{\alpha}(z)=\rho_{\delta}(z/r_{\alpha})\phi(z-d_{\alpha}) \nonumber \\
&\quad\phi(z)=\frac{a+b}{2}\frac{z+c}{\sqrt{1+(z+c)^2}}+\frac{a-b}{2}
\label{eq:phi}
\end{align}
where $0<a\le b$, $c=(b-a)/\sqrt{4ab}$. The integral of this function has a minimum at $z=d_{\alpha}=\|d\|_{\sigma}$ and a finite cut-off at $z=r_{\alpha}=\|r\|_{\sigma}$. 

\subsection{Flocking rules}

Based on the mathematical definition, we can derive the expression of Reynolds rules. 

The \emph{Flock Centering rule} is composed of two terms: one attractive and one repulsive term. This term are also called cohesion and separation terms. The cohesion term tend to move the agent towards the virtual center of mass of its neighbor. The separation term is responsible for avoiding collisions between neighbors and pushing them apart if they are too close one from each other.  These two terms are expressed together by the action function (Eq.~\ref{eq:phi}.
\begin{equation}
u_i^{mp}=\sum_{j\in N_i}{\phi_{\alpha}(\|q_j-q_i\|_{\sigma})}\vec{n}_{ij}
\label{eq:motionplanning}
\end{equation}
where $\phi_{\alpha}$ is the action function and $\vec{n}_{ij}=\sigma_{\epsilon}(q_j-q_i)$ is a vector connecting $q_i$ to $q_j$. 

The \emph{Velocity Matching} rule tends to align the velocity of the agent to the velocity of its neighbors. It is defined as 
\begin{equation}
u_i^{vm}=\sum_{j\in N_i}{a_{ij}(p_j-p_i)}
\label{eq:velocitymatching}
\end{equation}
where $p_i$ is the velocity of agent $i$ and the coefficients $a_{ij}=\rho_{\delta}(\|q_j-q_i\|_{\sigma}/r_{\sigma})$ are computed by the the bump function defined in Eq.~\ref{eq:bump}.


The \emph{Migration term} attracts the agent towards its goal and is defined by
\begin{equation}
u_i^{\gamma}=-c_1(q_i-q_g)-c_2\cdot p_i
\label{eq:migration}
\end{equation}  
where $c_1,c_2>0$ are constant, $q_i$ is the position of the agent and $q_g$ is the goal position. In the original formulation of this strategy, it was assumed that all agents had a common goal. In our work with personal aerial vehicles, the goal position is different for each user and therefore, the goal position can/is different for each agent. 

The final control input $u_i$ is composed of the sum of three term defined here above:
\begin{equation}
u_i=u_i^{mp}+u_i^{vm}+u_i^{\gamma}
\label{eq:input}
\end{equation}

\subsection{Optimization parameters}

The action function (Eq.\ref{eq:phi}) defines how agents interact with each other. The aim is too address the influence of the parameters of the action function on the objectives. The function has two parameters, namely $a$ and $b$ (note that $c$ is fully defined by $a$ and $b$). The individual effect of each parameter on the shape of the action function can be seen on Fig.\ref{fig:effects_actionCurve}

\begin{figure}[h]
    \centering
    \begin{subfigure}[b]{0.5\textwidth}
	    \setlength{\abovecaptionskip}{1pt plus 3pt minus 0pt}
	    % This file was created by matlab2tikz.
%
\begin{tikzpicture}

\begin{axis}[%
width=0.764\figW,
height=\figH,
at={(0\figW,0\figH)},
scale only axis,
every outer x axis line/.append style={black},
every x tick label/.append style={font=\color{black}},
xmin=0,
xmax=16,
xtick={ 0,  2,  4,  6,  8, 10, 12, 14, 16},
xlabel={$\text{$|$$|$q}_\text{j}\text{-q}_\text{i}\text{$|$$|$}$},
every outer y axis line/.append style={black},
every y tick label/.append style={font=\color{black}},
ymin=-9,
ymax=5,
ytick={-8, -6, -4, -2,  0,  2,  4},
axis background/.style={fill=white},
axis x line*=bottom,
axis y line*=left,
legend style={at={(1.03,1)},anchor=north west,legend cell align=left,align=left,draw=black},
xlabel shift={-4pt}
]
\addplot [color=red,solid]
  table[row sep=crcr]{%
0	-0\\
0.1	-0.299269979813303\\
0.2	-0.597643058396771\\
0.3	-0.89423566522595\\
0.4	-1.1881904311617\\
0.5	-1.47868816739967\\
0.6	-1.76495857491327\\
0.7	-2.04628938109857\\
0.8	-2.32203368864831\\
0.9	-2.59161541630041\\
1	-2.85453280457076\\
1.1	-3.11036004530868\\
1.2	-3.35874716678738\\
1.3	-3.59941836272638\\
1.4	-3.83216899265348\\
1.5	-4.05686150254159\\
1.6	-4.27342052024782\\
1.7	-4.48182737242136\\
1.8	-4.68211425123686\\
1.9	-4.87435823368657\\
2	-5.05867532619963\\
2.1	-5.23521467564571\\
2.2	-5.40415305642484\\
2.3	-5.565689713916\\
2.4	-5.7200416181107\\
2.5	-5.86743915840112\\
2.6	-6.00812229145813\\
2.7	-6.14233713887543\\
2.8	-6.27033301952284\\
2.9	-6.39235989297165\\
3	-6.50866618448829\\
3.1	-6.61949695848189\\
3.2	-6.72509240549032\\
3.3	-6.82548441921476\\
3.4	-6.91682840716877\\
3.5	-6.99739581724249\\
3.6	-7.06696004663902\\
3.7	-7.12532115863998\\
3.8	-7.17230728028945\\
3.9	-7.20777586691859\\
4	-7.23161482838137\\
4.1	-7.2437435124377\\
4.2	-7.24411354117508\\
4.3	-7.23270949669952\\
4.4	-7.20954945253926\\
4.5	-7.17468534727552\\
4.6	-7.12820319681719\\
4.7	-7.07022314143501\\
4.8	-7.00089932311255\\
4.9	-6.92041958788507\\
5	-6.8290050065197\\
5.1	-6.72690920500051\\
5.2	-6.61441749362011\\
5.3	-6.49184577976374\\
5.4	-6.35953924430114\\
5.5	-6.21787075429898\\
5.6	-6.06723897469503\\
5.7	-5.90806612740143\\
5.8	-5.74079532619918\\
5.9	-5.56588738698667\\
6	-5.38381697123676\\
6.1	-5.19506785936298\\
6.2	-5.00012705980656\\
6.3	-4.79947732257956\\
6.4	-4.59358741596941\\
6.5	-4.38289919784073\\
6.6	-4.16780999386659\\
6.7	-3.94864795625148\\
6.8	-3.72563669538053\\
6.9	-3.49884316013113\\
7	-3.2680987907865\\
7.1	-3.03287714411017\\
7.2	-2.79209938079827\\
7.3	-2.54381900913533\\
7.4	-2.28470616036701\\
7.5	-2.0092165859752\\
7.6	-1.70835597959428\\
7.7	-1.36834638738849\\
7.8	-0.971382891368651\\
7.9	-0.506207489026231\\
8	0\\
8.1	0.453554009433101\\
8.2	0.752629293324686\\
8.3	0.879755450057753\\
8.4	0.883854139704405\\
8.5	0.818498668673349\\
8.6	0.719579873848087\\
8.7	0.607837539496039\\
8.8	0.494933677043205\\
8.9	0.387571382825715\\
9	0.289767274919221\\
9.1	0.204045981388416\\
9.2	0.13206997843488\\
9.3	0.0749790876642189\\
9.4	0.0335787309657403\\
9.5	0.00844743638428441\\
9.6	0\\
9.7	0\\
9.8	0\\
9.9	0\\
10	0\\
10.1	0\\
10.2	0\\
10.3	0\\
10.4	0\\
10.5	0\\
10.6	0\\
10.7	0\\
10.8	0\\
10.9	0\\
11	0\\
11.1	0\\
11.2	0\\
11.3	0\\
11.4	0\\
11.5	0\\
};
\addlegendentry{d=8};

\addplot [color=blue,solid]
  table[row sep=crcr]{%
0	-0\\
0.1	-0.299528522991385\\
0.2	-0.598160545539665\\
0.3	-0.895012903830745\\
0.4	-1.18922864720793\\
0.5	-1.47998902282654\\
0.6	-1.76652419060389\\
0.7	-2.04812236513701\\
0.8	-2.32413716960111\\
0.9	-2.59399308128472\\
1	-2.85718894190541\\
1.1	-3.11329959160241\\
1.2	-3.36197575839647\\
1.3	-3.60294239160998\\
1.4	-3.83599566676511\\
1.5	-4.06099891101692\\
1.6	-4.27787770377923\\
1.7	-4.48661439934766\\
1.8	-4.68724230002036\\
1.9	-4.87983968260091\\
2	-5.06452385121079\\
2.1	-5.24144535763841\\
2.2	-5.41078249911034\\
2.3	-5.57273617395931\\
2.4	-5.72752514924123\\
2.5	-5.87538177152667\\
2.6	-6.01654813309575\\
2.7	-6.15127269055293\\
2.8	-6.27980732120194\\
2.9	-6.40240479401041\\
3	-6.51931662621179\\
3.1	-6.63079129308367\\
3.2	-6.73707275676515\\
3.3	-6.83839927972584\\
3.4	-6.93500248931646\\
3.5	-7.02710666141066\\
3.6	-7.11492819323542\\
3.7	-7.19867523788084\\
3.8	-7.27854747551985\\
3.9	-7.35470793877894\\
4	-7.42487372777091\\
4.1	-7.48746463863709\\
4.2	-7.54242999179877\\
4.3	-7.58972844876625\\
4.4	-7.62932834724765\\
4.5	-7.66120801141776\\
4.6	-7.6853560355767\\
4.7	-7.70177153974976\\
4.8	-7.71046439605667\\
4.9	-7.71145542491537\\
5	-7.70477656034584\\
5.1	-7.6904709838068\\
5.2	-7.6685932261351\\
5.3	-7.63920923726733\\
5.4	-7.60239642350644\\
5.5	-7.55824365215464\\
5.6	-7.50685122336834\\
5.7	-7.44833080909924\\
5.8	-7.38280535896901\\
5.9	-7.31040897287754\\
6	-7.23128674006506\\
6.1	-7.14559454422915\\
6.2	-7.05349883413071\\
6.3	-6.95517635889794\\
6.4	-6.85081386693846\\
6.5	-6.74060776697716\\
6.6	-6.62476374922418\\
6.7	-6.50349636400649\\
6.8	-6.37702855431856\\
6.9	-6.24559113759491\\
7	-6.10942223048574\\
7.1	-5.96876660839854\\
7.2	-5.82387498887069\\
7.3	-5.67500322420732\\
7.4	-5.52241138388909\\
7.5	-5.36636270050225\\
7.6	-5.20712234360245\\
7.7	-5.04495597286602\\
7.8	-4.880128003421\\
7.9	-4.71289948983074\\
8	-4.5435254969054\\
8.1	-4.37225176922386\\
8.2	-4.19931042727619\\
8.3	-4.0249142908797\\
8.4	-3.84924923439542\\
8.5	-3.67246367054386\\
8.6	-3.49465376754018\\
8.7	-3.31584220155749\\
8.8	-3.13594690974897\\
8.9	-2.95473403586483\\
9	-2.77174531633069\\
9.1	-2.58618319155126\\
9.2	-2.3967245068477\\
9.3	-2.20121164950207\\
9.4	-1.99613275241143\\
9.5	-1.77574995684846\\
9.6	-1.53071236815478\\
9.7	-1.24627361371349\\
9.8	-0.901959471552258\\
9.9	-0.480454389780927\\
10	-5.74441937126447e-15\\
10.1	0.450390555852559\\
10.2	0.764371901017294\\
10.3	0.917080471133783\\
10.4	0.951477073414525\\
10.5	0.916786535999093\\
10.6	0.8460466947438\\
10.7	0.758268247769631\\
10.8	0.664071376425615\\
10.9	0.569501792641379\\
11	0.478132914638606\\
11.1	0.392171963950856\\
11.2	0.313045091936783\\
11.3	0.241714388150343\\
11.4	0.1788548442931\\
11.5	0.124956233230472\\
11.6	0.0803835214364448\\
11.7	0.0454136907999905\\
11.8	0.0202587545144956\\
11.9	0.00508047863509588\\
12	0\\
12.1	0\\
12.2	0\\
12.3	0\\
12.4	0\\
12.5	0\\
12.6	0\\
12.7	0\\
12.8	0\\
12.9	0\\
13	0\\
13.1	0\\
13.2	0\\
13.3	0\\
13.4	0\\
13.5	0\\
13.6	0\\
13.7	0\\
13.8	0\\
13.9	0\\
14	0\\
14.1	0\\
14.2	0\\
14.3	0\\
14.4	0\\
};
\addlegendentry{d=10};

\addplot [color=green,solid]
  table[row sep=crcr]{%
0	-0\\
0.1	-0.299647681831648\\
0.2	-0.598398920018765\\
0.3	-0.895370609886431\\
0.4	-1.1897058645386\\
0.5	-1.48058600167814\\
0.6	-1.76724126055869\\
0.7	-2.0489599457032\\
0.8	-2.32509578237729\\
0.9	-2.59507336346152\\
1	-2.8583916608684\\
1.1	-3.1146256604141\\
1.2	-3.36342625195949\\
1.3	-3.60451856334581\\
1.4	-3.83769896568053\\
1.5	-4.06283099907186\\
1.6	-4.27984047351157\\
1.7	-4.4887099917525\\
1.8	-4.68947312272056\\
1.9	-4.88220842838242\\
2	-5.06703351702951\\
2.1	-5.244099264235\\
2.2	-5.41358431139603\\
2.3	-5.57568992235954\\
2.4	-5.73063525220569\\
2.5	-5.87865305943468\\
2.6	-6.01998587380583\\
2.7	-6.15488261686568\\
2.8	-6.2835956605265\\
2.9	-6.40637830054845\\
3	-6.52348261599892\\
3.1	-6.6351576822582\\
3.2	-6.74164810346857\\
3.3	-6.84319283007913\\
3.4	-6.94002422796288\\
3.5	-7.03236736716921\\
3.6	-7.12043950047189\\
3.7	-7.20444970427572\\
3.8	-7.28459865699687\\
3.9	-7.36107853261074\\
4	-7.43407298958207\\
4.1	-7.50375723779202\\
4.2	-7.5702981683183\\
4.3	-7.63385453298357\\
4.4	-7.69457716245383\\
4.5	-7.75250570813856\\
4.6	-7.80567461172651\\
4.7	-7.85329233163169\\
4.8	-7.89536532874662\\
4.9	-7.93190300104571\\
5	-7.96291785543279\\
5.1	-7.98842567032975\\
5.2	-8.00844564823009\\
5.3	-8.02300055762657\\
5.4	-8.03211686387744\\
5.5	-8.03582484870543\\
5.6	-8.03415871813146\\
5.7	-8.02715669873285\\
5.8	-8.01486112218825\\
5.9	-7.99731849812948\\
6	-7.97457957536646\\
6.1	-7.94669939158775\\
6.2	-7.91373731166654\\
6.3	-7.87575705472182\\
6.4	-7.83282671009819\\
6.5	-7.78501874243573\\
6.6	-7.73240998600424\\
6.7	-7.67508162847461\\
6.8	-7.61311918429404\\
6.9	-7.54661245782176\\
7	-7.47565549636714\\
7.1	-7.4003465332531\\
7.2	-7.32078792100329\\
7.3	-7.23708605472137\\
7.4	-7.14935128569397\\
7.5	-7.05769782520335\\
7.6	-6.96224363848119\\
7.7	-6.86311032866741\\
7.8	-6.76042301055554\\
7.9	-6.65431017380516\\
8	-6.54490353517655\\
8.1	-6.43233787918763\\
8.2	-6.31675088639972\\
8.3	-6.19828294829557\\
8.4	-6.07707696740672\\
8.5	-5.95327814095862\\
8.6	-5.8270337258052\\
8.7	-5.69849278178658\\
8.8	-5.56780588981841\\
8.9	-5.43512483994552\\
9	-5.30060228318162\\
9.1	-5.164391339089\\
9.2	-5.02664514856325\\
9.3	-4.88751635793743\\
9.4	-4.74715651597386\\
9.5	-4.60571535907731\\
9.6	-4.46333995143067\\
9.7	-4.3201736346642\\
9.8	-4.17635472454401\\
9.9	-4.03201486760083\\
10	-3.88727693491771\\
10.1	-3.74225227766663\\
10.2	-3.59703709022576\\
10.3	-3.45170750693114\\
10.4	-3.30631287319366\\
10.5	-3.16086633965829\\
10.6	-3.01533145875523\\
10.7	-2.86960269305542\\
10.8	-2.72347645418959\\
10.9	-2.57660707912217\\
11	-2.42843827584426\\
11.1	-2.27809364679319\\
11.2	-2.12419734681914\\
11.3	-1.96457315447497\\
11.4	-1.79573019950149\\
11.5	-1.61198175047604\\
11.6	-1.40399330646131\\
11.7	-1.15676505452971\\
11.8	-0.84864080896202\\
11.9	-0.458946231606757\\
12	0\\
12.1	0.443690089729577\\
12.2	0.764412500617351\\
12.3	0.932695233657069\\
12.4	0.987121613749085\\
12.5	0.973747875738061\\
12.6	0.923653976371904\\
12.7	0.854734464547454\\
12.8	0.776976174296139\\
12.9	0.696048440505536\\
13	0.615284164584707\\
13.1	0.536724337607174\\
13.2	0.461672138951829\\
13.3	0.390994175433836\\
13.4	0.325289325530241\\
13.5	0.264986392683904\\
13.6	0.210402310786876\\
13.7	0.161777822719098\\
13.8	0.119299919280333\\
13.9	0.0831162852462943\\
14	0.0533447992537326\\
14.1	0.0300799028836615\\
14.2	0.0133969468370054\\
14.3	0.00335520547982807\\
14.4	0\\
14.5	0\\
14.6	0\\
14.7	0\\
14.8	0\\
14.9	0\\
15	0\\
15.1	0\\
15.2	0\\
15.3	0\\
15.4	0\\
15.5	0\\
15.6	0\\
15.7	0\\
15.8	0\\
15.9	0\\
16	0\\
16.1	0\\
16.2	0\\
16.3	0\\
16.4	0\\
16.5	0\\
16.6	0\\
16.7	0\\
16.8	0\\
16.9	0\\
17	0\\
17.1	0\\
17.2	0\\
};
\addlegendentry{d=12};

\end{axis}
\end{tikzpicture}%
	    \caption{Influence of parameter d}
	\end{subfigure}
    \begin{subfigure}[b]{0.5\textwidth}
		\setlength{\abovecaptionskip}{1pt plus 3pt minus 0pt}	
	    % This file was created by matlab2tikz.
%
\definecolor{mycolor1}{rgb}{1.00000,1.00000,0.00000}%
\definecolor{mycolor2}{rgb}{1.00000,0.40000,0.69800}%
%
\begin{tikzpicture}

\begin{axis}[%
width=0.764\figW,
height=\figH,
at={(0\figW,0\figH)},
scale only axis,
every outer x axis line/.append style={black},
every x tick label/.append style={font=\color{black}},
xmin=0,
xmax=16,
xtick={ 0,  2,  4,  6,  8, 10, 12, 14, 16},
xlabel={$\text{$|$$|$q}_\text{j}\text{-q}_\text{i}\text{$|$$|$}$},
every outer y axis line/.append style={black},
every y tick label/.append style={font=\color{black}},
ymin=-9,
ymax=5,
ytick={-8, -6, -4, -2,  0,  2,  4},
axis background/.style={fill=white},
axis x line*=bottom,
axis y line*=left,
legend style={at={(1.03,1)},anchor=north west,legend cell align=left,align=left,draw=black},
xlabel shift={-4pt}
]
\addplot [color=red,solid]
  table[row sep=crcr]{%
0	-0\\
0.1	-0.299647681831648\\
0.2	-0.598398920018765\\
0.3	-0.895370609886431\\
0.4	-1.1897058645386\\
0.5	-1.48058600167814\\
0.6	-1.76724126055869\\
0.7	-2.0489599457032\\
0.8	-2.32509578237729\\
0.9	-2.59507336346152\\
1	-2.8583916608684\\
1.1	-3.1146256604141\\
1.2	-3.36342625195949\\
1.3	-3.60451856334581\\
1.4	-3.83769896568053\\
1.5	-4.06283099907186\\
1.6	-4.27984047351157\\
1.7	-4.4887099917525\\
1.8	-4.68947312272056\\
1.9	-4.88220842838242\\
2	-5.06703351702951\\
2.1	-5.244099264235\\
2.2	-5.41358431139603\\
2.3	-5.57568992235954\\
2.4	-5.73063525220569\\
2.5	-5.87865305943468\\
2.6	-6.01998587380583\\
2.7	-6.15488261686568\\
2.8	-6.2835956605265\\
2.9	-6.40637830054845\\
3	-6.52348261599892\\
3.1	-6.6351576822582\\
3.2	-6.74164810346857\\
3.3	-6.84319283007913\\
3.4	-6.94002422796288\\
3.5	-7.03236736716921\\
3.6	-7.12043950047189\\
3.7	-7.20444970427572\\
3.8	-7.28459865699687\\
3.9	-7.36107853261074\\
4	-7.43407298958207\\
4.1	-7.50375723779202\\
4.2	-7.5702981683183\\
4.3	-7.63385453298357\\
4.4	-7.69457716245383\\
4.5	-7.75250570813856\\
4.6	-7.80567461172651\\
4.7	-7.85329233163169\\
4.8	-7.89536532874662\\
4.9	-7.93190300104571\\
5	-7.96291785543279\\
5.1	-7.98842567032975\\
5.2	-8.00844564823009\\
5.3	-8.02300055762657\\
5.4	-8.03211686387744\\
5.5	-8.03582484870543\\
5.6	-8.03415871813146\\
5.7	-8.02715669873285\\
5.8	-8.01486112218825\\
5.9	-7.99731849812948\\
6	-7.97457957536646\\
6.1	-7.94669939158775\\
6.2	-7.91373731166654\\
6.3	-7.87575705472182\\
6.4	-7.83282671009819\\
6.5	-7.78501874243573\\
6.6	-7.73240998600424\\
6.7	-7.67508162847461\\
6.8	-7.61311918429404\\
6.9	-7.54661245782176\\
7	-7.47565549636714\\
7.1	-7.4003465332531\\
7.2	-7.32078792100329\\
7.3	-7.23708605472137\\
7.4	-7.14935128569397\\
7.5	-7.05769782520335\\
7.6	-6.96224363848119\\
7.7	-6.86311032866741\\
7.8	-6.76042301055554\\
7.9	-6.65431017380516\\
8	-6.54490353517655\\
8.1	-6.43233787918763\\
8.2	-6.31675088639972\\
8.3	-6.19828294829557\\
8.4	-6.07707696740672\\
8.5	-5.95327814095862\\
8.6	-5.8270337258052\\
8.7	-5.69849278178658\\
8.8	-5.56780588981841\\
8.9	-5.43512483994552\\
9	-5.30060228318162\\
9.1	-5.164391339089\\
9.2	-5.02664514856325\\
9.3	-4.88751635793743\\
9.4	-4.74715651597386\\
9.5	-4.60571535907731\\
9.6	-4.46333995143067\\
9.7	-4.3201736346642\\
9.8	-4.17635472454401\\
9.9	-4.03201486760083\\
10	-3.88727693491771\\
10.1	-3.74225227766663\\
10.2	-3.59703709022576\\
10.3	-3.45170750693114\\
10.4	-3.30631287319366\\
10.5	-3.16086633965829\\
10.6	-3.01533145875523\\
10.7	-2.86960269305542\\
10.8	-2.72347645418959\\
10.9	-2.57660707912217\\
11	-2.42843827584426\\
11.1	-2.27809364679319\\
11.2	-2.12419734681914\\
11.3	-1.96457315447497\\
11.4	-1.79573019950149\\
11.5	-1.61198175047604\\
11.6	-1.40399330646131\\
11.7	-1.15676505452971\\
11.8	-0.84864080896202\\
11.9	-0.458946231606757\\
12	0\\
12.1	0.443690089729577\\
12.2	0.764412500617351\\
12.3	0.932695233657069\\
12.4	0.987121613749085\\
12.5	0.973747875738061\\
12.6	0.923653976371904\\
12.7	0.854734464547454\\
12.8	0.776976174296139\\
12.9	0.696048440505536\\
13	0.615284164584707\\
13.1	0.536724337607174\\
13.2	0.461672138951829\\
13.3	0.390994175433836\\
13.4	0.325289325530241\\
13.5	0.264986392683904\\
13.6	0.210402310786876\\
13.7	0.161777822719098\\
13.8	0.119299919280333\\
13.9	0.0831162852462943\\
14	0.0533447992537326\\
14.1	0.0300799028836615\\
14.2	0.0133969468370054\\
14.3	0.00335520547982807\\
14.4	0\\
14.5	0\\
14.6	0\\
14.7	0\\
14.8	0\\
14.9	0\\
15	0\\
15.1	0\\
15.2	0\\
15.3	0\\
15.4	0\\
15.5	0\\
15.6	0\\
15.7	0\\
15.8	0\\
15.9	0\\
16	0\\
16.1	0\\
16.2	0\\
16.3	0\\
16.4	0\\
16.5	0\\
16.6	0\\
16.7	0\\
16.8	0\\
16.9	0\\
17	0\\
17.1	0\\
17.2	0\\
};
\addlegendentry{a=4};

\addplot [color=blue,solid]
  table[row sep=crcr]{%
0	-0\\
0.1	-0.299589844524595\\
0.2	-0.598283300583598\\
0.3	-0.895197317262439\\
0.4	-1.18947505863569\\
0.5	-1.48029788930779\\
0.6	-1.7668960902164\\
0.7	-2.0485580013641\\
0.8	-2.32463737648839\\
0.9	-2.59455882932614\\
1	-2.85782134462029\\
1.1	-3.11399991277212\\
1.2	-3.36274541993922\\
1.3	-3.60378298208448\\
1.4	-3.83690895050722\\
1.5	-4.06198683792849\\
1.6	-4.2789424198036\\
1.7	-4.48775825768159\\
1.8	-4.68846787312808\\
1.9	-4.88114977511056\\
2	-5.06592151378906\\
2.1	-5.24293390195357\\
2.2	-5.41236551400897\\
2.3	-5.57441754299696\\
2.4	-5.72930906972318\\
2.5	-5.87727277523158\\
2.6	-6.01855110887314\\
2.7	-6.15339290900534\\
2.8	-6.28205046168478\\
2.9	-6.40477697420749\\
3	-6.52182443457287\\
3.1	-6.63344182444229\\
3.2	-6.73987365149205\\
3.3	-6.84135876681499\\
3.4	-6.9381294338492\\
3.5	-7.03041061689845\\
3.6	-7.11841945940553\\
3.7	-7.20236492454239\\
3.8	-7.28244757323178\\
3.9	-7.3588594572942\\
4	-7.43178410793337\\
4.1	-7.50139660217356\\
4.2	-7.5678636921023\\
4.3	-7.63134398383072\\
4.4	-7.69198815494962\\
4.5	-7.74983573137581\\
4.6	-7.80292172848115\\
4.7	-7.85045481476582\\
4.8	-7.89244139083265\\
4.9	-7.92889079127046\\
5	-7.95981545617999\\
5.1	-7.98523109341437\\
5.2	-8.00515683075495\\
5.3	-8.01961535742824\\
5.4	-8.02863305452456\\
5.5	-8.03224011400841\\
5.6	-8.0304706461172\\
5.7	-8.023362775033\\
5.8	-8.01095872278276\\
5.9	-7.99330488138007\\
6	-7.97045187326652\\
6.1	-7.94245460014544\\
6.2	-7.90937228032716\\
6.3	-7.87126847472312\\
6.4	-7.82821110163782\\
6.5	-7.78027244051363\\
6.6	-7.72752912478347\\
6.7	-7.67006212398203\\
6.8	-7.60795671525646\\
6.9	-7.54130244440297\\
7	-7.470193076536\\
7.1	-7.39472653647162\\
7.2	-7.315004838875\\
7.3	-7.23113400818273\\
7.4	-7.14322398826362\\
7.5	-7.05138854172321\\
7.6	-6.95574513868686\\
7.7	-6.85641483480977\\
7.8	-6.75352213815678\\
7.9	-6.64719486446472\\
8	-6.53756398013934\\
8.1	-6.4247634321398\\
8.2	-6.30892996365451\\
8.3	-6.19020291416024\\
8.4	-6.06872400206253\\
8.5	-5.94463708761555\\
8.6	-5.81808791318123\\
8.7	-5.6892238170676\\
8.8	-5.55819341612441\\
8.9	-5.42514625089233\\
9	-5.29023238528656\\
9.1	-5.15360195039575\\
9.2	-5.01540461877571\\
9.3	-4.87578899131088\\
9.4	-4.73490187287028\\
9.5	-4.59288740497016\\
9.6	-4.44988601255542\\
9.7	-4.30603310646875\\
9.8	-4.16145746115762\\
9.9	-4.01627915558637\\
10	-3.870606919418\\
10.1	-3.72453465886539\\
10.2	-3.57813683535026\\
10.3	-3.43146221610998\\
10.4	-3.28452527761882\\
10.5	-3.13729416719397\\
10.6	-2.98967352473557\\
10.7	-2.84147947663177\\
10.8	-2.69240245444873\\
10.9	-2.54195064706773\\
11	-2.38936191314003\\
11.1	-2.23346307875809\\
11.2	-2.07243940761664\\
11.3	-1.90344774488018\\
11.4	-1.72195536124199\\
11.5	-1.52060701649686\\
11.6	-1.28735827035446\\
11.7	-1.00288154855644\\
11.8	-0.639291970262831\\
11.9	-0.169967964558642\\
12	0.389291466920896\\
12.1	0.929871715129662\\
12.2	1.31332371928776\\
12.3	1.50179099649174\\
12.4	1.5449156103176\\
12.5	1.50194645509963\\
12.6	1.41287597486155\\
12.7	1.300742672847\\
12.8	1.17841668027846\\
12.9	1.05321571295597\\
13	0.929452122148744\\
13.1	0.809775820935918\\
13.2	0.695886666347709\\
13.3	0.588921738048522\\
13.4	0.489672396660704\\
13.5	0.398709812598622\\
13.6	0.316459778003118\\
13.7	0.243248557217639\\
13.8	0.179331716696463\\
13.9	0.124912680167178\\
14	0.0801549262623705\\
14.1	0.0451901626875335\\
14.2	0.020123901379434\\
14.3	0.00503932345639077\\
14.4	0\\
14.5	0\\
14.6	0\\
14.7	0\\
14.8	0\\
14.9	0\\
15	0\\
15.1	0\\
15.2	0\\
15.3	0\\
15.4	0\\
15.5	0\\
15.6	0\\
15.7	0\\
15.8	0\\
15.9	0\\
16	0\\
16.1	0\\
16.2	0\\
16.3	0\\
16.4	0\\
16.5	0\\
16.6	0\\
16.7	0\\
16.8	0\\
16.9	0\\
17	0\\
17.1	0\\
17.2	0\\
};
\addlegendentry{a=6};

\addplot [color=green,solid]
  table[row sep=crcr]{%
0	-0\\
0.1	-0.299532007217543\\
0.2	-0.598167681148431\\
0.3	-0.895024024638448\\
0.4	-1.18924425273277\\
0.5	-1.48000977693744\\
0.6	-1.76655091987412\\
0.7	-2.048156057025\\
0.8	-2.32417897059949\\
0.9	-2.59404429519076\\
1	-2.85725102837219\\
1.1	-3.11337416513013\\
1.2	-3.36206458791894\\
1.3	-3.60304740082316\\
1.4	-3.8361189353339\\
1.5	-4.06114267678513\\
1.6	-4.27804436609562\\
1.7	-4.48680652361068\\
1.8	-4.6874626235356\\
1.9	-4.88009112183869\\
2	-5.0648095105486\\
2.1	-5.24176853967215\\
2.2	-5.41114671662192\\
2.3	-5.57314516363437\\
2.4	-5.72798288724066\\
2.5	-5.87589249102848\\
2.6	-6.01711634394045\\
2.7	-6.15190320114501\\
2.8	-6.28050526284305\\
2.9	-6.40317564786653\\
3	-6.52016625314682\\
3.1	-6.63172596662637\\
3.2	-6.73809919951553\\
3.3	-6.83952470355086\\
3.4	-6.93623463973552\\
3.5	-7.02845386662768\\
3.6	-7.11639941833917\\
3.7	-7.20028014480906\\
3.8	-7.2802964894667\\
3.9	-7.35664038197766\\
4	-7.42949522628468\\
4.1	-7.49903596655511\\
4.2	-7.5654292158863\\
4.3	-7.62883343467786\\
4.4	-7.68939914744541\\
4.5	-7.74716575461307\\
4.6	-7.8001688452358\\
4.7	-7.84761729789995\\
4.8	-7.88951745291868\\
4.9	-7.92587858149521\\
5	-7.95671305692719\\
5.1	-7.98203651649898\\
5.2	-8.0018680132798\\
5.3	-8.0162301572299\\
5.4	-8.02514924517168\\
5.5	-8.02865537931138\\
5.6	-8.02678257410294\\
5.7	-8.01956885133316\\
5.8	-8.00705632337726\\
5.9	-7.98929126463066\\
6	-7.96632417116659\\
6.1	-7.93820980870313\\
6.2	-7.90500724898778\\
6.3	-7.86677989472441\\
6.4	-7.82359549317745\\
6.5	-7.77552613859154\\
6.6	-7.72264826356269\\
6.7	-7.66504261948943\\
6.8	-7.60279424621888\\
6.9	-7.53599243098418\\
7	-7.46473065670485\\
7.1	-7.38910653969014\\
7.2	-7.30922175674672\\
7.3	-7.22518196164409\\
7.4	-7.13709669083326\\
7.5	-7.04507925824307\\
7.6	-6.94924663889253\\
7.7	-6.84971934095213\\
7.8	-6.74662126575803\\
7.9	-6.64007955512426\\
8	-6.53022442510213\\
8.1	-6.41718898509198\\
8.2	-6.3011090409093\\
8.3	-6.18212288002491\\
8.4	-6.06037103671835\\
8.5	-5.93599603427247\\
8.6	-5.80914210055727\\
8.7	-5.67995485234862\\
8.8	-5.54858094243042\\
8.9	-5.41516766183914\\
9	-5.27986248739151\\
9.1	-5.14281256170251\\
9.2	-5.00416408898819\\
9.3	-4.86406162468434\\
9.4	-4.72264722976672\\
9.5	-4.58005945086302\\
9.6	-4.43643207368017\\
9.7	-4.2918925782733\\
9.8	-4.14656019777124\\
9.9	-4.00054344357189\\
10	-3.8539369039183\\
10.1	-3.70681704006415\\
10.2	-3.55923658047476\\
10.3	-3.41121692528881\\
10.4	-3.26273768204397\\
10.5	-3.11372199472964\\
10.6	-2.9640155907159\\
10.7	-2.81335626020812\\
10.8	-2.66132845470787\\
10.9	-2.5072942150133\\
11	-2.35028555043579\\
11.1	-2.188832510723\\
11.2	-2.02068146841413\\
11.3	-1.84232233528538\\
11.4	-1.6481805229825\\
11.5	-1.42923228251768\\
11.6	-1.17072323424761\\
11.7	-0.848998042583185\\
11.8	-0.429943131563642\\
11.9	0.119010302489472\\
12	0.778582933841792\\
12.1	1.41605334052975\\
12.2	1.86223493795817\\
12.3	2.07088675932641\\
12.4	2.10270960688612\\
12.5	2.03014503446119\\
12.6	1.9020979733512\\
12.7	1.74675088114654\\
12.8	1.57985718626077\\
12.9	1.4103829854064\\
13	1.24362007971278\\
13.1	1.08282730426466\\
13.2	0.93010119374359\\
13.3	0.786849300663207\\
13.4	0.654055467791166\\
13.5	0.53243323251334\\
13.6	0.422517245219359\\
13.7	0.324719291716179\\
13.8	0.239363514112592\\
13.9	0.166709075088062\\
14	0.106965053271008\\
14.1	0.0603004224914054\\
14.2	0.0268508559218626\\
14.3	0.00672344143295346\\
14.4	0\\
14.5	0\\
14.6	0\\
14.7	0\\
14.8	0\\
14.9	0\\
15	0\\
15.1	0\\
15.2	0\\
15.3	0\\
15.4	0\\
15.5	0\\
15.6	0\\
15.7	0\\
15.8	0\\
15.9	0\\
16	0\\
16.1	0\\
16.2	0\\
16.3	0\\
16.4	0\\
16.5	0\\
16.6	0\\
16.7	0\\
16.8	0\\
16.9	0\\
17	0\\
17.1	0\\
17.2	0\\
};
\addlegendentry{a=8};

\addplot [color=mycolor1,solid]
  table[row sep=crcr]{%
0	-0\\
0.1	-0.29947416991049\\
0.2	-0.598052061713264\\
0.3	-0.894850732014457\\
0.4	-1.18901344682985\\
0.5	-1.47972166456708\\
0.6	-1.76620574953184\\
0.7	-2.0477541126859\\
0.8	-2.32372056471058\\
0.9	-2.59352976105537\\
1	-2.85668071212408\\
1.1	-3.11274841748815\\
1.2	-3.36138375589867\\
1.3	-3.60231181956183\\
1.4	-3.83532892016059\\
1.5	-4.06029851564177\\
1.6	-4.27714631238764\\
1.7	-4.48585478953978\\
1.8	-4.68645737394312\\
1.9	-4.87903246856683\\
2	-5.06369750730815\\
2.1	-5.24060317739072\\
2.2	-5.40992791923486\\
2.3	-5.57187278427179\\
2.4	-5.72665670475815\\
2.5	-5.87451220682538\\
2.6	-6.01568157900776\\
2.7	-6.15041349328467\\
2.8	-6.27896006400133\\
2.9	-6.40157432152558\\
3	-6.51850807172076\\
3.1	-6.63001010881046\\
3.2	-6.73632474753901\\
3.3	-6.83769064028672\\
3.4	-6.93433984562185\\
3.5	-7.02649711635691\\
3.6	-7.11437937727282\\
3.7	-7.19819536507573\\
3.8	-7.27814540570161\\
3.9	-7.35442130666112\\
4	-7.42720634463598\\
4.1	-7.49667533093666\\
4.2	-7.5629947396703\\
4.3	-7.62632288552501\\
4.4	-7.6868101399412\\
4.5	-7.74449577785032\\
4.6	-7.79741596199044\\
4.7	-7.84477978103409\\
4.8	-7.88659351500471\\
4.9	-7.92286637171997\\
5	-7.95361065767439\\
5.1	-7.9788419395836\\
5.2	-7.99857919580466\\
5.3	-8.01284495703156\\
5.4	-8.02166543581881\\
5.5	-8.02507064461436\\
5.6	-8.02309450208868\\
5.7	-8.01577492763332\\
5.8	-8.00315392397177\\
5.9	-7.98527764788125\\
6	-7.96219646906666\\
6.1	-7.93396501726083\\
6.2	-7.9006422176484\\
6.3	-7.86229131472571\\
6.4	-7.81897988471709\\
6.5	-7.77077983666944\\
6.6	-7.71776740234191\\
6.7	-7.66002311499685\\
6.8	-7.5976317771813\\
6.9	-7.53068241756539\\
7	-7.45926823687371\\
7.1	-7.38348654290866\\
7.2	-7.30343867461843\\
7.3	-7.21922991510545\\
7.4	-7.13096939340291\\
7.5	-7.03876997476294\\
7.6	-6.9427481390982\\
7.7	-6.84302384709449\\
7.8	-6.73972039335927\\
7.9	-6.63296424578382\\
8	-6.52288487006492\\
8.1	-6.40961453804415\\
8.2	-6.29328811816409\\
8.3	-6.17404284588958\\
8.4	-6.05201807137417\\
8.5	-5.9273549809294\\
8.6	-5.80019628793331\\
8.7	-5.67068588762964\\
8.8	-5.53896846873643\\
8.9	-5.40518907278594\\
9	-5.26949258949645\\
9.1	-5.13202317300926\\
9.2	-4.99292355920065\\
9.3	-4.85233425805779\\
9.4	-4.71039258666314\\
9.5	-4.56723149675588\\
9.6	-4.42297813480492\\
9.7	-4.27775205007786\\
9.8	-4.13166293438486\\
9.9	-3.98480773155742\\
10	-3.8372668884186\\
10.1	-3.68909942126291\\
10.2	-3.54033632559927\\
10.3	-3.39097163446765\\
10.4	-3.24095008646912\\
10.5	-3.09014982226532\\
10.6	-2.93835765669624\\
10.7	-2.78523304378447\\
10.8	-2.63025445496702\\
10.9	-2.47263778295886\\
11	-2.31120918773155\\
11.1	-2.14420194268791\\
11.2	-1.96892352921163\\
11.3	-1.78119692569059\\
11.4	-1.574405684723\\
11.5	-1.3378575485385\\
11.6	-1.05408819814076\\
11.7	-0.695114536609924\\
11.8	-0.220594292864453\\
11.9	0.407988569537587\\
12	1.16787440076269\\
12.1	1.90223496592983\\
12.2	2.41114615662858\\
12.3	2.63998252216108\\
12.4	2.66050360345464\\
12.5	2.55834361382276\\
12.6	2.39131997184085\\
12.7	2.19275908944608\\
12.8	1.98129769224309\\
12.9	1.76755025785684\\
13	1.55778803727682\\
13.1	1.35587878759341\\
13.2	1.16431572113947\\
13.3	0.984776863277893\\
13.4	0.818438538921628\\
13.5	0.666156652428059\\
13.6	0.528574712435601\\
13.7	0.40619002621472\\
13.8	0.299395311528722\\
13.9	0.208505470008945\\
14	0.133775180279646\\
14.1	0.0754106822952774\\
14.2	0.0335778104642912\\
14.3	0.00840755940951616\\
14.4	0\\
14.5	0\\
14.6	0\\
14.7	0\\
14.8	0\\
14.9	0\\
15	0\\
15.1	0\\
15.2	0\\
15.3	0\\
15.4	0\\
15.5	0\\
15.6	0\\
15.7	0\\
15.8	0\\
15.9	0\\
16	0\\
16.1	0\\
16.2	0\\
16.3	0\\
16.4	0\\
16.5	0\\
16.6	0\\
16.7	0\\
16.8	0\\
16.9	0\\
17	0\\
17.1	0\\
17.2	0\\
};
\addlegendentry{a=10};

\addplot [color=mycolor2,solid]
  table[row sep=crcr]{%
0	-0\\
0.1	-0.299416332603437\\
0.2	-0.597936442278097\\
0.3	-0.894677439390466\\
0.4	-1.18878264092693\\
0.5	-1.47943355219673\\
0.6	-1.76586057918955\\
0.7	-2.0473521683468\\
0.8	-2.32326215882168\\
0.9	-2.59301522691999\\
1	-2.85611039587597\\
1.1	-3.11212266984616\\
1.2	-3.3607029238784\\
1.3	-3.60157623830051\\
1.4	-3.83453890498727\\
1.5	-4.0594543544984\\
1.6	-4.27624825867966\\
1.7	-4.48490305546887\\
1.8	-4.68545212435063\\
1.9	-4.87797381529497\\
2	-5.06258550406769\\
2.1	-5.2394378151093\\
2.2	-5.4087091218478\\
2.3	-5.57060040490921\\
2.4	-5.72533052227564\\
2.5	-5.87313192262228\\
2.6	-6.01424681407507\\
2.7	-6.14892378542434\\
2.8	-6.27741486515961\\
2.9	-6.39997299518462\\
3	-6.51684989029471\\
3.1	-6.62829425099455\\
3.2	-6.73455029556249\\
3.3	-6.83585657702259\\
3.4	-6.93244505150817\\
3.5	-7.02454036608614\\
3.6	-7.11235933620646\\
3.7	-7.1961105853424\\
3.8	-7.27599432193653\\
3.9	-7.35220223134458\\
4	-7.42491746298728\\
4.1	-7.4943146953182\\
4.2	-7.5605602634543\\
4.3	-7.62381233637215\\
4.4	-7.68422113243699\\
4.5	-7.74182580108757\\
4.6	-7.79466307874509\\
4.7	-7.84194226416822\\
4.8	-7.88366957709074\\
4.9	-7.91985416194472\\
5	-7.95050825842159\\
5.1	-7.97564736266822\\
5.2	-7.99529037832952\\
5.3	-8.00945975683322\\
5.4	-8.01818162646593\\
5.5	-8.02148590991733\\
5.6	-8.01940643007443\\
5.7	-8.01198100393347\\
5.8	-7.99925152456627\\
5.9	-7.98126403113184\\
6	-7.95806876696672\\
6.1	-7.92972022581852\\
6.2	-7.89627718630902\\
6.3	-7.85780273472701\\
6.4	-7.81436427625672\\
6.5	-7.76603353474734\\
6.6	-7.71288654112113\\
6.7	-7.65500361050426\\
6.8	-7.59246930814373\\
6.9	-7.52537240414661\\
7	-7.45380581704257\\
7.1	-7.37786654612719\\
7.2	-7.29765559249015\\
7.3	-7.21327786856681\\
7.4	-7.12484209597256\\
7.5	-7.0324606912828\\
7.6	-6.93624963930388\\
7.7	-6.83632835323685\\
7.8	-6.73281952096051\\
7.9	-6.62584893644337\\
8	-6.51554531502771\\
8.1	-6.40204009099633\\
8.2	-6.28546719541888\\
8.3	-6.16596281175425\\
8.4	-6.04366510602999\\
8.5	-5.91871392758632\\
8.6	-5.79125047530935\\
8.7	-5.66141692291066\\
8.8	-5.52935599504243\\
8.9	-5.39521048373275\\
9	-5.2591226916014\\
9.1	-5.12123378431601\\
9.2	-4.98168302941313\\
9.3	-4.84060689143124\\
9.4	-4.69813794355957\\
9.5	-4.55440354264873\\
9.6	-4.40952419592966\\
9.7	-4.26361152188241\\
9.8	-4.11676567099847\\
9.9	-3.96907201954295\\
10	-3.8205968729189\\
10.1	-3.67138180246167\\
10.2	-3.52143607072377\\
10.3	-3.37072634364648\\
10.4	-3.21916249089427\\
10.5	-3.06657764980099\\
10.6	-2.91269972267658\\
10.7	-2.75710982736082\\
10.8	-2.59918045522616\\
10.9	-2.43798135090442\\
11	-2.27213282502731\\
11.1	-2.09957137465281\\
11.2	-1.91716559000912\\
11.3	-1.72007151609579\\
11.4	-1.50063084646351\\
11.5	-1.24648281455933\\
11.6	-0.937453162033904\\
11.7	-0.541231030636665\\
11.8	-0.0112454541652635\\
11.9	0.696966836585701\\
12	1.55716586768358\\
12.1	2.38841659132992\\
12.2	2.96005737529899\\
12.3	3.20907828499575\\
12.4	3.21829760002315\\
12.5	3.08654219318432\\
12.6	2.88054197033049\\
12.7	2.63876729774563\\
12.8	2.38273819822541\\
12.9	2.12471753030727\\
13	1.87195599484086\\
13.1	1.62893027092215\\
13.2	1.39853024853535\\
13.3	1.18270442589258\\
13.4	0.982821610052091\\
13.5	0.799880072342777\\
13.6	0.634632179651842\\
13.7	0.48766076071326\\
13.8	0.359427108944851\\
13.9	0.250301864929829\\
14	0.160585307288284\\
14.1	0.0905209420991494\\
14.2	0.0403047650067198\\
14.3	0.0100916773860789\\
14.4	0\\
14.5	0\\
14.6	0\\
14.7	0\\
14.8	0\\
14.9	0\\
15	0\\
15.1	0\\
15.2	0\\
15.3	0\\
15.4	0\\
15.5	0\\
15.6	0\\
15.7	0\\
15.8	0\\
15.9	0\\
16	0\\
16.1	0\\
16.2	0\\
16.3	0\\
16.4	0\\
16.5	0\\
16.6	0\\
16.7	0\\
16.8	0\\
16.9	0\\
17	0\\
17.1	0\\
17.2	0\\
};
\addlegendentry{a=12};

\end{axis}
\end{tikzpicture}%
	    \caption{Influence of parameter a}
	\end{subfigure}
	\begin{subfigure}[b]{0.5\textwidth}
		\setlength{\abovecaptionskip}{1pt plus 3pt minus 0pt}	
	    % This file was created by matlab2tikz.
%
\definecolor{mycolor1}{rgb}{1.00000,1.00000,0.00000}%
\definecolor{mycolor2}{rgb}{1.00000,0.40000,0.69800}%
\definecolor{mycolor3}{rgb}{0.37650,0.37650,0.37650}%
%
\begin{tikzpicture}

\begin{axis}[%
width=0.758\figW,
height=\figH,
at={(0\figW,0\figH)},
scale only axis,
every outer x axis line/.append style={black},
every x tick label/.append style={font=\color{black}},
xmin=0,
xmax=16,
xtick={ 0,  2,  4,  6,  8, 10, 12, 14, 16},
xlabel={$\text{$|$$|$q}_\text{j}\text{-q}_\text{i}\text{$|$$|$}$},
every outer y axis line/.append style={black},
every y tick label/.append style={font=\color{black}},
ymin=-9,
ymax=5,
ytick={-8, -6, -4, -2,  0,  2,  4},
axis background/.style={fill=white},
axis x line*=bottom,
axis y line*=left,
legend style={at={(1.03,1)},anchor=north west,legend cell align=left,align=left,draw=black},
xlabel shift={-4pt}
]
\addplot [color=red,solid]
  table[row sep=crcr]{%
0	-0\\
0.1	-0.0498448847935203\\
0.2	-0.0995404542778491\\
0.3	-0.148939613941086\\
0.4	-0.197899634251571\\
0.5	-0.2462841463291\\
0.6	-0.293964926189307\\
0.7	-0.340823417052034\\
0.8	-0.386751953914712\\
0.9	-0.431654670351282\\
1	-0.475448083064554\\
1.1	-0.518061363999042\\
1.2	-0.55943632195946\\
1.3	-0.599527125122094\\
1.4	-0.638299802324563\\
1.5	-0.675731564606369\\
1.6	-0.711809989405299\\
1.7	-0.746532108507238\\
1.8	-0.779903437799292\\
1.9	-0.811936982610633\\
2	-0.842652247437495\\
2.1	-0.872074273570123\\
2.2	-0.90023272292091\\
2.3	-0.927161021455618\\
2.4	-0.952895571230091\\
2.5	-0.977475036233949\\
2.6	-1.00093970407982\\
2.7	-1.02333092304372\\
2.8	-1.04469061201821\\
2.9	-1.06506083952314\\
3	-1.0844834669564\\
3.1	-1.10299985068318\\
3.2	-1.12065059728389\\
3.3	-1.13747536623963\\
3.4	-1.15351271447102\\
3.5	-1.16879997741025\\
3.6	-1.18337318163472\\
3.7	-1.1972669844904\\
3.8	-1.21051463655767\\
3.9	-1.22314796324089\\
4	-1.23519736218252\\
4.1	-1.24669181360125\\
4.2	-1.25765890102638\\
4.3	-1.2681248402425\\
4.4	-1.27811451456862\\
4.5	-1.28763432341852\\
4.6	-1.29635762987883\\
4.7	-1.3041528604955\\
4.8	-1.31102099160116\\
4.9	-1.31696348388221\\
5	-1.32198231048413\\
5.1	-1.32607998352932\\
5.2	-1.32925957891311\\
5.3	-1.33152475927386\\
5.4	-1.33287979505811\\
5.5	-1.33332958362253\\
5.6	-1.33287966633148\\
5.7	-1.3315362436224\\
5.8	-1.32930618802222\\
5.9	-1.32619705510589\\
6	-1.32221709239452\\
6.1	-1.31737524619411\\
6.2	-1.31168116637879\\
6.3	-1.30514520912246\\
6.4	-1.29777843758242\\
6.5	-1.28959262053579\\
6.6	-1.28060022896608\\
6.7	-1.27081443059145\\
6.8	-1.26024908231971\\
6.9	-1.24891872060565\\
7	-1.23683854967595\\
7.1	-1.22402442757305\\
7.2	-1.21049284995341\\
7.3	-1.19626093155583\\
7.4	-1.18134638523174\\
7.5	-1.16576749840033\\
7.6	-1.14954310675632\\
7.7	-1.13269256501517\\
7.8	-1.11523571442799\\
7.9	-1.09719284673345\\
8	-1.07858466413407\\
8.1	-1.0594322347849\\
8.2	-1.03975694315794\\
8.3	-1.01958043449038\\
8.4	-0.998924552327483\\
8.5	-0.977811267921314\\
8.6	-0.956262599927597\\
8.7	-0.934300522432796\\
8.8	-0.91194685881308\\
8.9	-0.889223158235598\\
9	-0.866150550705178\\
9.1	-0.842749575359423\\
9.2	-0.819039975114657\\
9.3	-0.795040448611989\\
9.4	-0.770768347489691\\
9.5	-0.746239303000979\\
9.6	-0.721466760446358\\
9.7	-0.696461392118288\\
9.8	-0.671230348446696\\
9.9	-0.645776291242689\\
10	-0.620096129986782\\
10.1	-0.594179348275706\\
10.2	-0.568005756911797\\
10.3	-0.541542433119914\\
10.4	-0.514739486240866\\
10.5	-0.487524102502506\\
10.6	-0.459792019759767\\
10.7	-0.431395088136485\\
10.8	-0.402122742796837\\
10.9	-0.371673793096302\\
11	-0.339612441466982\\
11.1	-0.305297994407041\\
11.2	-0.26776965913235\\
11.3	-0.225553176421172\\
11.4	-0.176330302817756\\
11.5	-0.116372401780708\\
11.6	-0.039607157565465\\
11.7	0.0636783342004828\\
11.8	0.207474596338312\\
11.9	0.405139406479065\\
12	0.648819111534827\\
12.1	0.884251057288404\\
12.2	1.04225411455357\\
12.3	1.1039421436673\\
12.4	1.09417692990571\\
12.5	1.04262227822562\\
12.6	0.969312326878063\\
12.7	0.885802757923814\\
12.8	0.798563539019884\\
12.9	0.711286860834978\\
13	0.626160623370846\\
13.1	0.544539861815769\\
13.2	0.467302902151773\\
13.3	0.395044966930115\\
13.4	0.328186672805811\\
13.5	0.267036765305181\\
13.6	0.211829497158215\\
13.7	0.162747527950751\\
13.8	0.119936315573605\\
13.9	0.0835133724091884\\
14	0.0535743448900186\\
14.1	0.0301970834870636\\
14.2	0.0134444153768819\\
14.3	0.00336606420757584\\
14.4	0\\
14.5	0\\
14.6	0\\
14.7	0\\
14.8	0\\
14.9	0\\
15	0\\
15.1	0\\
15.2	0\\
15.3	0\\
15.4	0\\
15.5	0\\
15.6	0\\
15.7	0\\
15.8	0\\
15.9	0\\
16	0\\
16.1	0\\
16.2	0\\
16.3	0\\
16.4	0\\
16.5	0\\
16.6	0\\
16.7	0\\
16.8	0\\
16.9	0\\
17	0\\
17.1	0\\
17.2	0\\
};
\addlegendentry{b=0.5};

\addplot [color=blue,solid]
  table[row sep=crcr]{%
0	-0\\
0.1	-0.0998054442011458\\
0.2	-0.199312147426032\\
0.3	-0.298225813130155\\
0.4	-0.396260880308978\\
0.5	-0.493144517398909\\
0.6	-0.588620193063184\\
0.7	-0.682450722782267\\
0.8	-0.774420719607228\\
0.9	-0.86433840897333\\
1	-0.952036798625323\\
1.1	-1.03737422328205\\
1.2	-1.12023430795947\\
1.3	-1.20052541276684\\
1.4	-1.27817963499576\\
1.5	-1.35315145149947\\
1.6	-1.42541608622655\\
1.7	-1.49496768515629\\
1.8	-1.56181737478354\\
1.9	-1.62599127176499\\
2	-1.6875285013559\\
2.1	-1.7464792717031\\
2.2	-1.80290304061593\\
2.3	-1.8568668016364\\
2.4	-1.90844350742521\\
2.5	-1.95771064087409\\
2.6	-2.00474893802502\\
2.7	-2.04964126180811\\
2.8	-2.09247162171987\\
2.9	-2.13332433172821\\
3	-2.1722832967649\\
3.1	-2.20943141699818\\
3.2	-2.24485009852083\\
3.3	-2.27861885900753\\
3.4	-2.31081501716939\\
3.5	-2.34151345536205\\
3.6	-2.37078644540215\\
3.7	-2.39870352844747\\
3.8	-2.42533144064551\\
3.9	-2.45073407711486\\
4	-2.47497248766243\\
4.1	-2.4981048984394\\
4.2	-2.52018675448477\\
4.3	-2.54127077879072\\
4.4	-2.56140704414566\\
4.5	-2.58060860036253\\
4.6	-2.59822102624836\\
4.7	-2.61398075472274\\
4.8	-2.62788985903025\\
4.9	-2.63995138731491\\
5	-2.65016941947386\\
5.1	-2.65854912088941\\
5.2	-2.66509679277651\\
5.3	-2.66981991894441\\
5.4	-2.67272720882198\\
5.5	-2.67382863663911\\
5.6	-2.67313547669148\\
5.7	-2.67066033464449\\
5.8	-2.66641717485542\\
5.9	-2.66042134371061\\
6	-2.65268958898891\\
6.1	-2.64324007527284\\
6.2	-2.63209239543634\\
6.3	-2.61926757824234\\
6.4	-2.60478809208557\\
6.5	-2.58867784491578\\
6.6	-2.57096218037371\\
6.7	-2.55166787016809\\
6.8	-2.53082310271457\\
6.9	-2.50845746804887\\
7	-2.48460193901419\\
7.1	-2.45928884870906\\
7.2	-2.43255186416338\\
7.3	-2.40442595618894\\
7.4	-2.37494736532419\\
7.5	-2.34415356376094\\
7.6	-2.31208321310129\\
7.7	-2.27877611774562\\
7.8	-2.2442731736535\\
7.9	-2.20861631214779\\
8	-2.17184843834257\\
8.1	-2.13401336366544\\
8.2	-2.09515573180629\\
8.3	-2.05532093725142\\
8.4	-2.01455503534333\\
8.5	-1.97290464252877\\
8.6	-1.93041682510312\\
8.7	-1.88713897430355\\
8.8	-1.84311866501414\\
8.9	-1.79840349457758\\
9	-1.75304089720047\\
9.1	-1.70707792810534\\
9.2	-1.66056100980438\\
9.3	-1.61353563047708\\
9.4	-1.56604598118652\\
9.5	-1.51813451421624\\
9.6	-1.46984139864322\\
9.7	-1.42120384062747\\
9.8	-1.37225522366616\\
9.9	-1.32302400651432\\
10	-1.27353229097297\\
10.1	-1.22379393415389\\
10.2	-1.17381202357459\\
10.3	-1.12357544788216\\
10.4	-1.07305416363142\\
10.5	-1.02219254993366\\
10.6	-0.970899907558859\\
10.7	-0.919036609120272\\
10.8	-0.866393485075387\\
10.9	-0.812660450301474\\
11	-0.757377608342438\\
11.1	-0.69985712488427\\
11.2	-0.639055196669708\\
11.3	-0.573357172031932\\
11.4	-0.500210282154502\\
11.5	-0.415494271519775\\
11.6	-0.312484387344635\\
11.7	-0.180410343545555\\
11.8	-0.00374848472175446\\
11.9	0.232322278861901\\
12	0.519055289227861\\
12.1	0.796138863776639\\
12.2	0.98668579176633\\
12.3	1.06969276166525\\
12.4	1.07276586667438\\
12.5	1.02884739772811\\
12.6	0.960180656776831\\
12.7	0.879589099248542\\
12.8	0.794246066075135\\
12.9	0.70823917676909\\
13	0.623985331613618\\
13.1	0.54297675697405\\
13.2	0.466176749511784\\
13.3	0.394234808630859\\
13.4	0.327607203350697\\
13.5	0.266626690780926\\
13.6	0.211544059883947\\
13.7	0.16255358690442\\
13.8	0.11980903631495\\
13.9	0.0834339549766096\\
14	0.0535284357627614\\
14.1	0.0301736473663831\\
14.2	0.0134349216689066\\
14.3	0.00336389246202628\\
14.4	0\\
14.5	0\\
14.6	0\\
14.7	0\\
14.8	0\\
14.9	0\\
15	0\\
15.1	0\\
15.2	0\\
15.3	0\\
15.4	0\\
15.5	0\\
15.6	0\\
15.7	0\\
15.8	0\\
15.9	0\\
16	0\\
16.1	0\\
16.2	0\\
16.3	0\\
16.4	0\\
16.5	0\\
16.6	0\\
16.7	0\\
16.8	0\\
16.9	0\\
17	0\\
17.1	0\\
17.2	0\\
};
\addlegendentry{b=1.0};

\addplot [color=green,solid]
  table[row sep=crcr]{%
0	-0\\
0.1	-0.149766003608771\\
0.2	-0.299083840574215\\
0.3	-0.447512012319224\\
0.4	-0.594622126366385\\
0.5	-0.740004888468718\\
0.6	-0.88327545993706\\
0.7	-1.0240780285125\\
0.8	-1.16208948529974\\
0.9	-1.29702214759538\\
1	-1.42862551418609\\
1.1	-1.55668708256507\\
1.2	-1.68103229395947\\
1.3	-1.80152370041158\\
1.4	-1.91805946766695\\
1.5	-2.03057133839256\\
1.6	-2.13902218304781\\
1.7	-2.24340326180534\\
1.8	-2.3437313117678\\
1.9	-2.44004556091935\\
2	-2.5324047552743\\
2.1	-2.62088426983607\\
2.2	-2.70557335831096\\
2.3	-2.78657258181719\\
2.4	-2.86399144362033\\
2.5	-2.93794624551424\\
2.6	-3.00855817197022\\
2.7	-3.0759516005725\\
2.8	-3.14025263142153\\
2.9	-3.20158782393327\\
3	-3.26008312657341\\
3.1	-3.31586298331319\\
3.2	-3.36904959975777\\
3.3	-3.41976235177543\\
3.4	-3.46811731986776\\
3.5	-3.51422693331384\\
3.6	-3.55819970916959\\
3.7	-3.60014007240453\\
3.8	-3.64014824473335\\
3.9	-3.67832019098883\\
4	-3.71474761314234\\
4.1	-3.74951798327756\\
4.2	-3.78271460794315\\
4.3	-3.81441671733893\\
4.4	-3.8446995737227\\
4.5	-3.87358287730653\\
4.6	-3.9000844226179\\
4.7	-3.92380864894998\\
4.8	-3.94475872645934\\
4.9	-3.96293929074761\\
5	-3.9783565284636\\
5.1	-3.99101825824949\\
5.2	-4.0009340066399\\
5.3	-4.00811507861495\\
5.4	-4.01257462258584\\
5.5	-4.01432768965569\\
5.6	-4.01339128705147\\
5.7	-4.00978442566658\\
5.8	-4.00352816168863\\
5.9	-3.99464563231533\\
6	-3.98316208558329\\
6.1	-3.96910490435157\\
6.2	-3.95250362449389\\
6.3	-3.93338994736221\\
6.4	-3.91179774658873\\
6.5	-3.88776306929577\\
6.6	-3.86132413178135\\
6.7	-3.83252130974472\\
6.8	-3.80139712310944\\
6.9	-3.76799621549209\\
7	-3.73236532835243\\
7.1	-3.69455326984507\\
7.2	-3.65461087837336\\
7.3	-3.61259098082204\\
7.4	-3.56854834541663\\
7.5	-3.52253962912154\\
7.6	-3.47462331944627\\
7.7	-3.42485967047607\\
7.8	-3.37331063287901\\
7.9	-3.32003977756213\\
8	-3.26511221255106\\
8.1	-3.20859449254599\\
8.2	-3.15055452045465\\
8.3	-3.09106144001245\\
8.4	-3.03018551835918\\
8.5	-2.96799801713624\\
8.6	-2.90457105027864\\
8.7	-2.83997742617431\\
8.8	-2.77429047121521\\
8.9	-2.70758383091957\\
9	-2.63993124369575\\
9.1	-2.57140628085125\\
9.2	-2.50208204449409\\
9.3	-2.43203081234217\\
9.4	-2.36132361488336\\
9.5	-2.29002972543151\\
9.6	-2.21821603684008\\
9.7	-2.14594628913665\\
9.8	-2.07328009888562\\
9.9	-2.00027172178595\\
10	-1.92696845195915\\
10.1	-1.85340852003208\\
10.2	-1.77961829023738\\
10.3	-1.70560846264441\\
10.4	-1.63136884102198\\
10.5	-1.55686099736482\\
10.6	-1.48200779535795\\
10.7	-1.40667813010406\\
10.8	-1.33066422735394\\
10.9	-1.25364710750665\\
11	-1.17514277521789\\
11.1	-1.0944162553615\\
11.2	-1.01034073420707\\
11.3	-0.921161167642691\\
11.4	-0.824090261491249\\
11.5	-0.714616141258842\\
11.6	-0.585361617123804\\
11.7	-0.424499021291593\\
11.8	-0.214971565781821\\
11.9	0.059505151244736\\
12	0.389291466920896\\
12.1	0.708026670264874\\
12.2	0.931117468979085\\
12.3	1.03544337966321\\
12.4	1.05135480344306\\
12.5	1.0150725172306\\
12.6	0.951048986675599\\
12.7	0.87337544057327\\
12.8	0.789928593130386\\
12.9	0.705191492703201\\
13	0.62181003985639\\
13.1	0.541413652132331\\
13.2	0.465050596871795\\
13.3	0.393424650331604\\
13.4	0.327027733895583\\
13.5	0.26621661625667\\
13.6	0.21125862260968\\
13.7	0.16235964585809\\
13.8	0.119681757056296\\
13.9	0.0833545375440308\\
14	0.0534825266355042\\
14.1	0.0301502112457027\\
14.2	0.0134254279609313\\
14.3	0.00336172071647673\\
14.4	0\\
14.5	0\\
14.6	0\\
14.7	0\\
14.8	0\\
14.9	0\\
15	0\\
15.1	0\\
15.2	0\\
15.3	0\\
15.4	0\\
15.5	0\\
15.6	0\\
15.7	0\\
15.8	0\\
15.9	0\\
16	0\\
16.1	0\\
16.2	0\\
16.3	0\\
16.4	0\\
16.5	0\\
16.6	0\\
16.7	0\\
16.8	0\\
16.9	0\\
17	0\\
17.1	0\\
17.2	0\\
};
\addlegendentry{b=1.5};

\addplot [color=mycolor1,solid]
  table[row sep=crcr]{%
0	-0\\
0.1	-0.199726563016397\\
0.2	-0.398855533722398\\
0.3	-0.596798211508293\\
0.4	-0.792983372423791\\
0.5	-0.986865259538527\\
0.6	-1.17793072681094\\
0.7	-1.36570533424273\\
0.8	-1.54975825099226\\
0.9	-1.72970588621743\\
1	-1.90521422974686\\
1.1	-2.07599994184808\\
1.2	-2.24183027995948\\
1.3	-2.40252198805632\\
1.4	-2.55793930033815\\
1.5	-2.70799122528566\\
1.6	-2.85262827986906\\
1.7	-2.99183883845439\\
1.8	-3.12564524875205\\
1.9	-3.2540998500737\\
2	-3.3772810091927\\
2.1	-3.49528926796905\\
2.2	-3.60824367600598\\
2.3	-3.71627836199797\\
2.4	-3.81953937981545\\
2.5	-3.91818185015439\\
2.6	-4.01236740591543\\
2.7	-4.1022619393369\\
2.8	-4.18803364112319\\
2.9	-4.26985131613833\\
3	-4.34788295638191\\
3.1	-4.42229454962819\\
3.2	-4.4932491009947\\
3.3	-4.56090584454333\\
3.4	-4.62541962256613\\
3.5	-4.68694041126563\\
3.6	-4.74561297293702\\
3.7	-4.80157661636159\\
3.8	-4.85496504882119\\
3.9	-4.9059063048628\\
4	-4.95452273862225\\
4.1	-5.00093106811571\\
4.2	-5.04524246140153\\
4.3	-5.08756265588715\\
4.4	-5.12799210329974\\
4.5	-5.16655715425054\\
4.6	-5.20194781898744\\
4.7	-5.23363654317721\\
4.8	-5.26162759388843\\
4.9	-5.28592719418031\\
5	-5.30654363745333\\
5.1	-5.32348739560958\\
5.2	-5.3367712205033\\
5.3	-5.34641023828549\\
5.4	-5.35242203634971\\
5.5	-5.35482674267227\\
5.6	-5.35364709741147\\
5.7	-5.34890851668867\\
5.8	-5.34063914852184\\
5.9	-5.32886992092005\\
6	-5.31363458217768\\
6.1	-5.29496973343029\\
6.2	-5.27291485355144\\
6.3	-5.24751231648208\\
6.4	-5.21880740109188\\
6.5	-5.18684829367576\\
6.6	-5.15168608318898\\
6.7	-5.11337474932135\\
6.8	-5.07197114350431\\
6.9	-5.02753496293531\\
7	-4.98012871769067\\
7.1	-4.92981769098108\\
7.2	-4.87666989258333\\
7.3	-4.82075600545515\\
7.4	-4.76214932550908\\
7.5	-4.70092569448214\\
7.6	-4.63716342579124\\
7.7	-4.57094322320651\\
7.8	-4.50234809210452\\
7.9	-4.43146324297648\\
8	-4.35837598675956\\
8.1	-4.28317562142654\\
8.2	-4.20595330910301\\
8.3	-4.12680194277349\\
8.4	-4.04581600137502\\
8.5	-3.9630913917437\\
8.6	-3.87872527545416\\
8.7	-3.79281587804507\\
8.8	-3.70546227741628\\
8.9	-3.61676416726155\\
9	-3.52682159019104\\
9.1	-3.43573463359717\\
9.2	-3.34360307918381\\
9.3	-3.25052599420726\\
9.4	-3.15660124858019\\
9.5	-3.06192493664678\\
9.6	-2.96659067503695\\
9.7	-2.87068873764583\\
9.8	-2.77430497410508\\
9.9	-2.67751943705758\\
10	-2.58040461294534\\
10.1	-2.48302310591026\\
10.2	-2.38542455690017\\
10.3	-2.28764147740665\\
10.4	-2.18968351841254\\
10.5	-2.09152944479598\\
10.6	-1.99311568315704\\
10.7	-1.89431965108785\\
10.8	-1.79493496963249\\
10.9	-1.69463376471182\\
11	-1.59290794209335\\
11.1	-1.48897538583873\\
11.2	-1.38162627174442\\
11.3	-1.26896516325345\\
11.4	-1.147970240828\\
11.5	-1.01373801099791\\
11.6	-0.858238846902973\\
11.7	-0.66858769903763\\
11.8	-0.426194646841887\\
11.9	-0.113311976372428\\
12	0.259527644613931\\
12.1	0.619914476753108\\
12.2	0.87554914619184\\
12.3	1.00119399766116\\
12.4	1.02994374021173\\
12.5	1.00129763673308\\
12.6	0.941917316574367\\
12.7	0.867161781897998\\
12.8	0.785611120185637\\
12.9	0.702143808637313\\
13	0.619634748099163\\
13.1	0.539850547290612\\
13.2	0.463924444231806\\
13.3	0.392614492032348\\
13.4	0.326448264440469\\
13.5	0.265806541732415\\
13.6	0.210973185335412\\
13.7	0.162165704811759\\
13.8	0.119554477797642\\
13.9	0.083275120111452\\
14	0.053436617508247\\
14.1	0.0301267751250223\\
14.2	0.013415934252956\\
14.3	0.00335954897092718\\
14.4	0\\
14.5	0\\
14.6	0\\
14.7	0\\
14.8	0\\
14.9	0\\
15	0\\
15.1	0\\
15.2	0\\
15.3	0\\
15.4	0\\
15.5	0\\
15.6	0\\
15.7	0\\
15.8	0\\
15.9	0\\
16	0\\
16.1	0\\
16.2	0\\
16.3	0\\
16.4	0\\
16.5	0\\
16.6	0\\
16.7	0\\
16.8	0\\
16.9	0\\
17	0\\
17.1	0\\
17.2	0\\
};
\addlegendentry{b=2.0};

\addplot [color=mycolor2,solid]
  table[row sep=crcr]{%
0	-0\\
0.1	-0.249687122424022\\
0.2	-0.498627226870582\\
0.3	-0.746084410697362\\
0.4	-0.991344618481198\\
0.5	-1.23372563060834\\
0.6	-1.47258599368481\\
0.7	-1.70733263997296\\
0.8	-1.93742701668477\\
0.9	-2.16238962483947\\
1	-2.38180294530763\\
1.1	-2.59531280113109\\
1.2	-2.80262826595948\\
1.3	-3.00352027570107\\
1.4	-3.19781913300934\\
1.5	-3.38541111217876\\
1.6	-3.56623437669032\\
1.7	-3.74027441510345\\
1.8	-3.9075591857363\\
1.9	-4.06815413922806\\
2	-4.22215726311111\\
2.1	-4.36969426610202\\
2.2	-4.51091399370101\\
2.3	-4.64598414217875\\
2.4	-4.77508731601057\\
2.5	-4.89841745479453\\
2.6	-5.01617663986063\\
2.7	-5.12857227810129\\
2.8	-5.23581465082485\\
2.9	-5.33811480834339\\
3	-5.43568278619042\\
3.1	-5.5287261159432\\
3.2	-5.61744860223164\\
3.3	-5.70204933731123\\
3.4	-5.7827219252645\\
3.5	-5.85965388921742\\
3.6	-5.93302623670445\\
3.7	-6.00301316031866\\
3.8	-6.06978185290903\\
3.9	-6.13349241873677\\
4	-6.19429786410216\\
4.1	-6.25234415295386\\
4.2	-6.30777031485992\\
4.3	-6.36070859443536\\
4.4	-6.41128463287679\\
4.5	-6.45953143119455\\
4.6	-6.50381121535697\\
4.7	-6.54346443740445\\
4.8	-6.57849646131753\\
4.9	-6.60891509761301\\
5	-6.63473074644306\\
5.1	-6.65595653296966\\
5.2	-6.67260843436669\\
5.3	-6.68470539795603\\
5.4	-6.69226945011357\\
5.5	-6.69532579568885\\
5.6	-6.69390290777146\\
5.7	-6.68803260771076\\
5.8	-6.67775013535504\\
5.9	-6.66309420952477\\
6	-6.64410707877207\\
6.1	-6.62083456250902\\
6.2	-6.59332608260899\\
6.3	-6.56163468560195\\
6.4	-6.52581705559504\\
6.5	-6.48593351805574\\
6.6	-6.44204803459661\\
6.7	-6.39422818889798\\
6.8	-6.34254516389918\\
6.9	-6.28707371037854\\
7	-6.2278921070289\\
7.1	-6.16508211211709\\
7.2	-6.09872890679331\\
7.3	-6.02892103008826\\
7.4	-5.95575030560152\\
7.5	-5.87931175984274\\
7.6	-5.79970353213621\\
7.7	-5.71702677593696\\
7.8	-5.63138555133003\\
7.9	-5.54288670839082\\
8	-5.45163976096806\\
8.1	-5.35775675030708\\
8.2	-5.26135209775136\\
8.3	-5.16254244553453\\
8.4	-5.06144648439087\\
8.5	-4.95818476635116\\
8.6	-4.85287950062968\\
8.7	-4.74565432991583\\
8.8	-4.63663408361734\\
8.9	-4.52594450360354\\
9	-4.41371193668633\\
9.1	-4.30006298634308\\
9.2	-4.18512411387353\\
9.3	-4.06902117607235\\
9.4	-3.95187888227702\\
9.5	-3.83382014786204\\
9.6	-3.71496531323381\\
9.7	-3.59543118615502\\
9.8	-3.47532984932454\\
9.9	-3.35476715232921\\
10	-3.23384077393152\\
10.1	-3.11263769178845\\
10.2	-2.99123082356297\\
10.3	-2.8696744921689\\
10.4	-2.7479981958031\\
10.5	-2.62619789222714\\
10.6	-2.50422357095614\\
10.7	-2.38196117207163\\
10.8	-2.25920571191104\\
10.9	-2.13562042191699\\
11	-2.01067310896881\\
11.1	-1.88353451631596\\
11.2	-1.75291180928178\\
11.3	-1.61676915886421\\
11.4	-1.47185022016474\\
11.5	-1.31285988073697\\
11.6	-1.13111607668214\\
11.7	-0.912676376783668\\
11.8	-0.637417727901954\\
11.9	-0.286129103989593\\
12	0.129763822306965\\
12.1	0.531802283241342\\
12.2	0.819980823404596\\
12.3	0.966944615659115\\
12.4	1.00853267698041\\
12.5	0.987522756235573\\
12.6	0.932785646473136\\
12.7	0.860948123222726\\
12.8	0.781293647240888\\
12.9	0.699096124571425\\
13	0.617459456341935\\
13.1	0.538287442448893\\
13.2	0.462798291591817\\
13.3	0.391804333733092\\
13.4	0.325868794985355\\
13.5	0.265396467208159\\
13.6	0.210687748061144\\
13.7	0.161971763765429\\
13.8	0.119427198538987\\
13.9	0.0831957026788731\\
14	0.0533907083809898\\
14.1	0.0301033390043419\\
14.2	0.0134064405449807\\
14.3	0.00335737722537762\\
14.4	0\\
14.5	0\\
14.6	0\\
14.7	0\\
14.8	0\\
14.9	0\\
15	0\\
15.1	0\\
15.2	0\\
15.3	0\\
15.4	0\\
15.5	0\\
15.6	0\\
15.7	0\\
15.8	0\\
15.9	0\\
16	0\\
16.1	0\\
16.2	0\\
16.3	0\\
16.4	0\\
16.5	0\\
16.6	0\\
16.7	0\\
16.8	0\\
16.9	0\\
17	0\\
17.1	0\\
17.2	0\\
};
\addlegendentry{b=2.5};

\addplot [color=mycolor3,solid]
  table[row sep=crcr]{%
0	-0\\
0.1	-0.299647681831648\\
0.2	-0.598398920018765\\
0.3	-0.895370609886431\\
0.4	-1.1897058645386\\
0.5	-1.48058600167814\\
0.6	-1.76724126055869\\
0.7	-2.0489599457032\\
0.8	-2.32509578237729\\
0.9	-2.59507336346152\\
1	-2.8583916608684\\
1.1	-3.1146256604141\\
1.2	-3.36342625195949\\
1.3	-3.60451856334581\\
1.4	-3.83769896568053\\
1.5	-4.06283099907186\\
1.6	-4.27984047351157\\
1.7	-4.4887099917525\\
1.8	-4.68947312272056\\
1.9	-4.88220842838242\\
2	-5.06703351702951\\
2.1	-5.244099264235\\
2.2	-5.41358431139603\\
2.3	-5.57568992235954\\
2.4	-5.73063525220569\\
2.5	-5.87865305943468\\
2.6	-6.01998587380583\\
2.7	-6.15488261686568\\
2.8	-6.2835956605265\\
2.9	-6.40637830054845\\
3	-6.52348261599892\\
3.1	-6.6351576822582\\
3.2	-6.74164810346857\\
3.3	-6.84319283007913\\
3.4	-6.94002422796288\\
3.5	-7.03236736716921\\
3.6	-7.12043950047189\\
3.7	-7.20444970427572\\
3.8	-7.28459865699687\\
3.9	-7.36107853261074\\
4	-7.43407298958207\\
4.1	-7.50375723779202\\
4.2	-7.5702981683183\\
4.3	-7.63385453298357\\
4.4	-7.69457716245383\\
4.5	-7.75250570813856\\
4.6	-7.80567461172651\\
4.7	-7.85329233163169\\
4.8	-7.89536532874662\\
4.9	-7.93190300104571\\
5	-7.96291785543279\\
5.1	-7.98842567032975\\
5.2	-8.00844564823009\\
5.3	-8.02300055762657\\
5.4	-8.03211686387744\\
5.5	-8.03582484870543\\
5.6	-8.03415871813146\\
5.7	-8.02715669873285\\
5.8	-8.01486112218825\\
5.9	-7.99731849812948\\
6	-7.97457957536646\\
6.1	-7.94669939158775\\
6.2	-7.91373731166654\\
6.3	-7.87575705472182\\
6.4	-7.83282671009819\\
6.5	-7.78501874243573\\
6.6	-7.73240998600424\\
6.7	-7.67508162847461\\
6.8	-7.61311918429404\\
6.9	-7.54661245782176\\
7	-7.47565549636714\\
7.1	-7.4003465332531\\
7.2	-7.32078792100329\\
7.3	-7.23708605472137\\
7.4	-7.14935128569397\\
7.5	-7.05769782520335\\
7.6	-6.96224363848119\\
7.7	-6.86311032866741\\
7.8	-6.76042301055554\\
7.9	-6.65431017380516\\
8	-6.54490353517655\\
8.1	-6.43233787918763\\
8.2	-6.31675088639972\\
8.3	-6.19828294829557\\
8.4	-6.07707696740672\\
8.5	-5.95327814095862\\
8.6	-5.8270337258052\\
8.7	-5.69849278178658\\
8.8	-5.56780588981841\\
8.9	-5.43512483994552\\
9	-5.30060228318162\\
9.1	-5.164391339089\\
9.2	-5.02664514856325\\
9.3	-4.88751635793743\\
9.4	-4.74715651597386\\
9.5	-4.60571535907731\\
9.6	-4.46333995143067\\
9.7	-4.3201736346642\\
9.8	-4.17635472454401\\
9.9	-4.03201486760083\\
10	-3.88727693491771\\
10.1	-3.74225227766663\\
10.2	-3.59703709022576\\
10.3	-3.45170750693114\\
10.4	-3.30631287319366\\
10.5	-3.16086633965829\\
10.6	-3.01533145875523\\
10.7	-2.86960269305542\\
10.8	-2.72347645418959\\
10.9	-2.57660707912217\\
11	-2.42843827584426\\
11.1	-2.27809364679319\\
11.2	-2.12419734681914\\
11.3	-1.96457315447497\\
11.4	-1.79573019950149\\
11.5	-1.61198175047604\\
11.6	-1.40399330646131\\
11.7	-1.15676505452971\\
11.8	-0.84864080896202\\
11.9	-0.458946231606757\\
12	0\\
12.1	0.443690089729577\\
12.2	0.764412500617351\\
12.3	0.932695233657069\\
12.4	0.987121613749085\\
12.5	0.973747875738061\\
12.6	0.923653976371904\\
12.7	0.854734464547454\\
12.8	0.776976174296139\\
12.9	0.696048440505536\\
13	0.615284164584707\\
13.1	0.536724337607174\\
13.2	0.461672138951829\\
13.3	0.390994175433836\\
13.4	0.325289325530241\\
13.5	0.264986392683904\\
13.6	0.210402310786876\\
13.7	0.161777822719098\\
13.8	0.119299919280333\\
13.9	0.0831162852462943\\
14	0.0533447992537326\\
14.1	0.0300799028836615\\
14.2	0.0133969468370054\\
14.3	0.00335520547982807\\
14.4	0\\
14.5	0\\
14.6	0\\
14.7	0\\
14.8	0\\
14.9	0\\
15	0\\
15.1	0\\
15.2	0\\
15.3	0\\
15.4	0\\
15.5	0\\
15.6	0\\
15.7	0\\
15.8	0\\
15.9	0\\
16	0\\
16.1	0\\
16.2	0\\
16.3	0\\
16.4	0\\
16.5	0\\
16.6	0\\
16.7	0\\
16.8	0\\
16.9	0\\
17	0\\
17.1	0\\
17.2	0\\
};
\addlegendentry{b=3.0};

\end{axis}
\end{tikzpicture}%
	    \caption{Influence of parameter b}
	\end{subfigure}

    \caption{Influence of the parameters on the action function}
    \label{fig:effects_actionCurve}
\end{figure} 
